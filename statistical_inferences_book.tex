% Options for packages loaded elsewhere
\PassOptionsToPackage{unicode}{hyperref}
\PassOptionsToPackage{hyphens}{url}
\PassOptionsToPackage{dvipsnames,svgnames,x11names}{xcolor}
%
\documentclass[
  oneside]{krantz}
\usepackage{amsmath,amssymb}
\usepackage{lmodern}
\usepackage{iftex}
\ifPDFTeX
  \usepackage[T1]{fontenc}
  \usepackage[utf8]{inputenc}
  \usepackage{textcomp} % provide euro and other symbols
\else % if luatex or xetex
  \usepackage{unicode-math}
  \defaultfontfeatures{Scale=MatchLowercase}
  \defaultfontfeatures[\rmfamily]{Ligatures=TeX,Scale=1}
\fi
% Use upquote if available, for straight quotes in verbatim environments
\IfFileExists{upquote.sty}{\usepackage{upquote}}{}
\IfFileExists{microtype.sty}{% use microtype if available
  \usepackage[]{microtype}
  \UseMicrotypeSet[protrusion]{basicmath} % disable protrusion for tt fonts
}{}
\makeatletter
\@ifundefined{KOMAClassName}{% if non-KOMA class
  \IfFileExists{parskip.sty}{%
    \usepackage{parskip}
  }{% else
    \setlength{\parindent}{0pt}
    \setlength{\parskip}{6pt plus 2pt minus 1pt}}
}{% if KOMA class
  \KOMAoptions{parskip=half}}
\makeatother
\usepackage{xcolor}
\usepackage[margin=1in]{geometry}
\usepackage{graphicx}
\makeatletter
\def\maxwidth{\ifdim\Gin@nat@width>\linewidth\linewidth\else\Gin@nat@width\fi}
\def\maxheight{\ifdim\Gin@nat@height>\textheight\textheight\else\Gin@nat@height\fi}
\makeatother
% Scale images if necessary, so that they will not overflow the page
% margins by default, and it is still possible to overwrite the defaults
% using explicit options in \includegraphics[width, height, ...]{}
\setkeys{Gin}{width=\maxwidth,height=\maxheight,keepaspectratio}
% Set default figure placement to htbp
\makeatletter
\def\fps@figure{htbp}
\makeatother
\setlength{\emergencystretch}{3em} % prevent overfull lines
\providecommand{\tightlist}{%
  \setlength{\itemsep}{0pt}\setlength{\parskip}{0pt}}
\setcounter{secnumdepth}{-\maxdimen} % remove section numbering
\newlength{\cslhangindent}
\setlength{\cslhangindent}{1.5em}
\newlength{\csllabelwidth}
\setlength{\csllabelwidth}{3em}
\newlength{\cslentryspacingunit} % times entry-spacing
\setlength{\cslentryspacingunit}{\parskip}
\newenvironment{CSLReferences}[2] % #1 hanging-ident, #2 entry spacing
 {% don't indent paragraphs
  \setlength{\parindent}{0pt}
  % turn on hanging indent if param 1 is 1
  \ifodd #1
  \let\oldpar\par
  \def\par{\hangindent=\cslhangindent\oldpar}
  \fi
  % set entry spacing
  \setlength{\parskip}{#2\cslentryspacingunit}
 }%
 {}
\usepackage{calc}
\newcommand{\CSLBlock}[1]{#1\hfill\break}
\newcommand{\CSLLeftMargin}[1]{\parbox[t]{\csllabelwidth}{#1}}
\newcommand{\CSLRightInline}[1]{\parbox[t]{\linewidth - \csllabelwidth}{#1}\break}
\newcommand{\CSLIndent}[1]{\hspace{\cslhangindent}#1}
\usepackage{booktabs}
\usepackage{longtable}
\usepackage{array}
\usepackage{multirow}
\usepackage{wrapfig}
\usepackage{float}
\usepackage{colortbl}
\usepackage{pdflscape}
\usepackage{tabu}
\usepackage{threeparttable}
\usepackage{threeparttablex}
\usepackage[normalem]{ulem}
\usepackage{makecell}
\usepackage{xcolor}
\ifLuaTeX
  \usepackage{selnolig}  % disable illegal ligatures
\fi
\IfFileExists{bookmark.sty}{\usepackage{bookmark}}{\usepackage{hyperref}}
\IfFileExists{xurl.sty}{\usepackage{xurl}}{} % add URL line breaks if available
\urlstyle{same} % disable monospaced font for URLs
\hypersetup{
  pdftitle={Improving Your Statistical Inferences},
  pdfauthor={Daniël Lakens},
  colorlinks=true,
  linkcolor={Maroon},
  filecolor={Maroon},
  citecolor={Blue},
  urlcolor={Blue},
  pdfcreator={LaTeX via pandoc}}

\title{Improving Your Statistical Inferences}
\author{Daniël Lakens}
\date{2022-09-02}

\begin{document}
\maketitle

\listoffigures
\listoftables
\hypertarget{introduction}{%
\section*{Introduction}\label{introduction}}
\addcontentsline{toc}{section}{Introduction}

Placeholder

\hypertarget{pvalue}{%
\section{\texorpdfstring{Using \emph{p}-values to test a
hypothesis}{Using p-values to test a hypothesis}}\label{pvalue}}

Placeholder

\hypertarget{philosophical-approaches-to-p-values}{%
\subsection{\texorpdfstring{Philosophical approaches to
\emph{p}-values}{Philosophical approaches to p-values}}\label{philosophical-approaches-to-p-values}}

\hypertarget{creating-a-null-model}{%
\subsection{Creating a null model}\label{creating-a-null-model}}

\hypertarget{calculating-a-p-value}{%
\subsection{\texorpdfstring{Calculating a
\emph{p}-value}{Calculating a p-value}}\label{calculating-a-p-value}}

\hypertarget{whichpexpect}{%
\subsection{\texorpdfstring{Which \emph{p}-values can you
expect?}{Which p-values can you expect?}}\label{whichpexpect}}

\hypertarget{lindley}{%
\subsection{Lindley's paradox}\label{lindley}}

\hypertarget{correctly-reporting-and-interpreting-p-values}{%
\subsection{\texorpdfstring{Correctly reporting and interpreting
\emph{p}-values}{Correctly reporting and interpreting p-values}}\label{correctly-reporting-and-interpreting-p-values}}

\hypertarget{misconceptions}{%
\subsection{\texorpdfstring{Preventing common misconceptions about
\emph{p}-values}{Preventing common misconceptions about p-values}}\label{misconceptions}}

\hypertarget{misconception1}{%
\subsubsection{\texorpdfstring{Misunderstanding 1: A non-significant
\emph{p}-value means that the null hypothesis is
true.}{Misunderstanding 1: A non-significant p-value means that the null hypothesis is true.}}\label{misconception1}}

\hypertarget{misunderstanding-2-a-significant-p-value-means-that-the-null-hypothesis-is-false.}{%
\subsubsection{\texorpdfstring{Misunderstanding 2: A significant
\emph{p}-value means that the null hypothesis is
false.}{Misunderstanding 2: A significant p-value means that the null hypothesis is false.}}\label{misunderstanding-2-a-significant-p-value-means-that-the-null-hypothesis-is-false.}}

\hypertarget{misunderstanding-3-a-significant-p-value-means-that-a-practically-important-effect-has-been-discovered.}{%
\subsubsection{\texorpdfstring{Misunderstanding 3: A significant
\emph{p}-value means that a practically important effect has been
discovered.}{Misunderstanding 3: A significant p-value means that a practically important effect has been discovered.}}\label{misunderstanding-3-a-significant-p-value-means-that-a-practically-important-effect-has-been-discovered.}}

\hypertarget{misconception4}{%
\subsubsection{Misunderstanding 4: If you have observed a significant
finding, the probability that you have made a Type 1 error (a false
positive) is 5\%.}\label{misconception4}}

\hypertarget{misunderstanding-5-one-minus-the-p-value-is-the-probability-that-the-effect-will-replicate-when-repeated.}{%
\subsubsection{\texorpdfstring{Misunderstanding 5: One minus the
\emph{p}-value is the probability that the effect will replicate when
repeated.}{Misunderstanding 5: One minus the p-value is the probability that the effect will replicate when repeated.}}\label{misunderstanding-5-one-minus-the-p-value-is-the-probability-that-the-effect-will-replicate-when-repeated.}}

\hypertarget{test-yourself}{%
\subsection{Test Yourself}\label{test-yourself}}

\hypertarget{questions-about-which-p-values-you-can-expect}{%
\subsubsection{\texorpdfstring{Questions about which \emph{p}-values you
can
expect}{Questions about which p-values you can expect}}\label{questions-about-which-p-values-you-can-expect}}

\hypertarget{questions-about-p-value-misconceptions}{%
\subsubsection{\texorpdfstring{Questions about \emph{p}-value
misconceptions}{Questions about p-value misconceptions}}\label{questions-about-p-value-misconceptions}}

\hypertarget{open-questions}{%
\subsubsection{Open Questions}\label{open-questions}}

\hypertarget{errorcontrol}{%
\section{Error control}\label{errorcontrol}}

Placeholder

\hypertarget{which-outcome-can-you-expect-if-you-perform-a-study}{%
\subsection{Which outcome can you expect if you perform a
study?}\label{which-outcome-can-you-expect-if-you-perform-a-study}}

\hypertarget{ppv}{%
\subsection{Positive predictive value}\label{ppv}}

\hypertarget{type-1-error-inflation}{%
\subsection{Type 1 error inflation}\label{type-1-error-inflation}}

\hypertarget{optionalstopping}{%
\subsection{Optional stopping}\label{optionalstopping}}

\hypertarget{justifyerrorrate}{%
\subsection{Justifying Error Rates}\label{justifyerrorrate}}

\hypertarget{why-you-dont-need-to-adjust-your-alpha-level-for-all-tests-youll-do-in-your-lifetime.}{%
\subsection{Why you don't need to adjust your alpha level for all tests
you'll do in your
lifetime.}\label{why-you-dont-need-to-adjust-your-alpha-level-for-all-tests-youll-do-in-your-lifetime.}}

\hypertarget{power-analysis}{%
\subsection{Power Analysis}\label{power-analysis}}

\hypertarget{test-yourself-1}{%
\subsection{Test Yourself}\label{test-yourself-1}}

\hypertarget{questions-about-the-positive-predictive-value}{%
\subsubsection{Questions about the positive predictive
value}\label{questions-about-the-positive-predictive-value}}

\hypertarget{questions-about-optional-stopping}{%
\subsubsection{Questions about optional
stopping}\label{questions-about-optional-stopping}}

\hypertarget{open-questions-1}{%
\subsubsection{Open Questions}\label{open-questions-1}}

\hypertarget{likelihoods}{%
\section{Likelihoods}\label{likelihoods}}

Placeholder

\hypertarget{likelihood-ratios}{%
\subsection{Likelihood ratios}\label{likelihood-ratios}}

\hypertarget{likelihood-of-mixed-results-in-sets-of-studies}{%
\subsection{Likelihood of mixed results in sets of
studies}\label{likelihood-of-mixed-results-in-sets-of-studies}}

\hypertarget{likettest}{%
\subsection{\texorpdfstring{Likelihoods for
\emph{t}-tests}{Likelihoods for t-tests}}\label{likettest}}

\hypertarget{test-yourself-2}{%
\subsection{Test Yourself}\label{test-yourself-2}}

\hypertarget{questions-about-likelihoods}{%
\subsubsection{Questions about
likelihoods}\label{questions-about-likelihoods}}

\hypertarget{questions-about-mixed-results}{%
\subsubsection{Questions about mixed
results}\label{questions-about-mixed-results}}

\hypertarget{open-questions-2}{%
\subsubsection{Open Questions}\label{open-questions-2}}

\hypertarget{bayes}{%
\section{Bayesian statistics}\label{bayes}}

Placeholder

\hypertarget{bayes-factors}{%
\subsection{Bayes factors}\label{bayes-factors}}

\hypertarget{updating-our-belief}{%
\subsection{Updating our belief}\label{updating-our-belief}}

\hypertarget{bayesest}{%
\subsection{Bayesian Estimation}\label{bayesest}}

\hypertarget{test-yourself-3}{%
\subsection{Test Yourself}\label{test-yourself-3}}

\hypertarget{open-questions-3}{%
\subsubsection{Open Questions}\label{open-questions-3}}

\hypertarget{questions}{%
\section{Asking Statistical Questions}\label{questions}}

Placeholder

\hypertarget{description}{%
\subsection{Description}\label{description}}

\hypertarget{prediction}{%
\subsection{Prediction}\label{prediction}}

\hypertarget{explanation}{%
\subsection{Explanation}\label{explanation}}

\hypertarget{loosening-and-tightening}{%
\subsection{Loosening and Tightening}\label{loosening-and-tightening}}

\hypertarget{three-statistical-philosophies}{%
\subsection{Three statistical
philosophies}\label{three-statistical-philosophies}}

\hypertarget{do-you-really-want-to-test-a-hypothesis}{%
\subsection{Do You Really Want to Test a
Hypothesis?}\label{do-you-really-want-to-test-a-hypothesis}}

\hypertarget{onesided}{%
\subsection{Directional (One-Sided) versus Non-Directional (Two-Sided)
Tests}\label{onesided}}

\hypertarget{crud}{%
\subsection{Systematic Noise, or the Crud Factor}\label{crud}}

\hypertarget{effectsize}{%
\section{Effect Sizes}\label{effectsize}}

Placeholder

\hypertarget{effect-sizes}{%
\subsection{Effect sizes}\label{effect-sizes}}

\hypertarget{the-facebook-experiment}{%
\subsection{The Facebook experiment}\label{the-facebook-experiment}}

\hypertarget{the-hungry-judges-study}{%
\subsection{The Hungry Judges study}\label{the-hungry-judges-study}}

\hypertarget{cohend}{%
\subsection{Standardised Mean Differences}\label{cohend}}

\hypertarget{interpreting-effect-sizes}{%
\subsection{Interpreting effect sizes}\label{interpreting-effect-sizes}}

\hypertarget{correlations-and-variance-explained}{%
\subsection{Correlations and Variance
Explained}\label{correlations-and-variance-explained}}

\hypertarget{correcting-for-bias}{%
\subsection{Correcting for Bias}\label{correcting-for-bias}}

\hypertarget{effect-sizes-for-interactions}{%
\subsection{Effect Sizes for
Interactions}\label{effect-sizes-for-interactions}}

\hypertarget{test-yourself-4}{%
\subsection{Test Yourself}\label{test-yourself-4}}

\hypertarget{open-questions-4}{%
\subsubsection{Open Questions}\label{open-questions-4}}

\hypertarget{confint}{%
\section{Confidence Intervals}\label{confint}}

Placeholder

\hypertarget{population-vs.-sample}{%
\subsection{Population vs.~Sample}\label{population-vs.-sample}}

\hypertarget{what-is-a-confidence-interval}{%
\subsection{What is a Confidence
Interval?}\label{what-is-a-confidence-interval}}

\hypertarget{relatCIp}{%
\subsection{\texorpdfstring{The relation between confidence intervals
and
\emph{p}-values}{The relation between confidence intervals and p-values}}\label{relatCIp}}

\hypertarget{the-standard-error-and-95-confidence-intervals}{%
\subsection{The Standard Error and 95\% Confidence
Intervals}\label{the-standard-error-and-95-confidence-intervals}}

\hypertarget{overlapping-confidence-intervals}{%
\subsection{Overlapping Confidence
Intervals}\label{overlapping-confidence-intervals}}

\hypertarget{prediction-intervals}{%
\subsection{Prediction Intervals}\label{prediction-intervals}}

\hypertarget{capture-percentages}{%
\subsection{Capture Percentages}\label{capture-percentages}}

\hypertarget{calculating-confidence-intervals-around-standard-deviations.}{%
\subsection{Calculating Confidence Intervals around Standard
Deviations.}\label{calculating-confidence-intervals-around-standard-deviations.}}

\hypertarget{computing-confidence-intervals-around-effect-sizes}{%
\subsection{Computing Confidence Intervals around Effect
Sizes}\label{computing-confidence-intervals-around-effect-sizes}}

\hypertarget{test-yourself-5}{%
\subsection{Test Yourself}\label{test-yourself-5}}

\hypertarget{open-questions-5}{%
\subsubsection{Open Questions}\label{open-questions-5}}

\hypertarget{power}{%
\section{Sample Size Justification}\label{power}}

Placeholder

\hypertarget{six-approaches-to-justify-sample-sizes}{%
\subsection{Six Approaches to Justify Sample
Sizes}\label{six-approaches-to-justify-sample-sizes}}

\hypertarget{six-ways-to-evaluate-which-effect-sizes-are-interesting}{%
\subsection{Six Ways to Evaluate Which Effect Sizes are
Interesting}\label{six-ways-to-evaluate-which-effect-sizes-are-interesting}}

\hypertarget{the-value-of-information}{%
\subsection{The Value of Information}\label{the-value-of-information}}

\hypertarget{measuring-almost-the-entire-population}{%
\subsection{Measuring (Almost) the Entire
Population}\label{measuring-almost-the-entire-population}}

\hypertarget{resource-constraints}{%
\subsection{Resource Constraints}\label{resource-constraints}}

\hypertarget{aprioripower}{%
\subsection{A-priori Power Analysis}\label{aprioripower}}

\hypertarget{planning-for-precision}{%
\subsection{Planning for Precision}\label{planning-for-precision}}

\hypertarget{heuristics}{%
\subsection{Heuristics}\label{heuristics}}

\hypertarget{no-justification}{%
\subsection{No Justification}\label{no-justification}}

\hypertarget{what-is-your-inferential-goal}{%
\subsection{What is Your Inferential
Goal?}\label{what-is-your-inferential-goal}}

\hypertarget{what-is-the-smallest-effect-size-of-interest}{%
\subsection{What is the Smallest Effect Size of
Interest?}\label{what-is-the-smallest-effect-size-of-interest}}

\hypertarget{minimaldetectable}{%
\subsection{The Minimal Statistically Detectable
Effect}\label{minimaldetectable}}

\hypertarget{what-is-the-expected-effect-size}{%
\subsection{What is the Expected Effect
Size?}\label{what-is-the-expected-effect-size}}

\hypertarget{using-an-estimate-from-a-meta-analysis}{%
\subsection{Using an Estimate from a
Meta-Analysis}\label{using-an-estimate-from-a-meta-analysis}}

\hypertarget{using-an-estimate-from-a-previous-study}{%
\subsection{Using an Estimate from a Previous
Study}\label{using-an-estimate-from-a-previous-study}}

\hypertarget{using-an-estimate-from-a-theoretical-model}{%
\subsection{Using an Estimate from a Theoretical
Model}\label{using-an-estimate-from-a-theoretical-model}}

\hypertarget{compute-the-width-of-the-confidence-interval-around-the-effect-size}{%
\subsection{Compute the Width of the Confidence Interval around the
Effect
Size}\label{compute-the-width-of-the-confidence-interval-around-the-effect-size}}

\hypertarget{plot-a-sensitivity-power-analysis}{%
\subsection{Plot a Sensitivity Power
Analysis}\label{plot-a-sensitivity-power-analysis}}

\hypertarget{the-distribution-of-effect-sizes-in-a-research-area}{%
\subsection{The Distribution of Effect Sizes in a Research
Area}\label{the-distribution-of-effect-sizes-in-a-research-area}}

\hypertarget{additional-considerations-when-designing-an-informative-study}{%
\subsection{Additional Considerations When Designing an Informative
Study}\label{additional-considerations-when-designing-an-informative-study}}

\hypertarget{compromise-power-analysis}{%
\subsection{Compromise Power Analysis}\label{compromise-power-analysis}}

\hypertarget{posthocpower}{%
\subsection{What to do if Your Editor Asks for Post-hoc
Power?}\label{posthocpower}}

\hypertarget{sequentialsamplesize}{%
\subsection{Sequential Analyses}\label{sequentialsamplesize}}

\hypertarget{increasing-power-without-increasing-the-sample-size}{%
\subsection{Increasing Power Without Increasing the Sample
Size}\label{increasing-power-without-increasing-the-sample-size}}

\hypertarget{know-your-measure}{%
\subsection{Know Your Measure}\label{know-your-measure}}

\hypertarget{conventions-as-meta-heuristics}{%
\subsection{Conventions as
meta-heuristics}\label{conventions-as-meta-heuristics}}

\hypertarget{sample-size-justification-in-qualitative-research}{%
\subsection{Sample Size Justification in Qualitative
Research}\label{sample-size-justification-in-qualitative-research}}

\hypertarget{discussion}{%
\subsection{Discussion}\label{discussion}}

\hypertarget{test-yourself-6}{%
\subsection{Test Yourself}\label{test-yourself-6}}

\hypertarget{open-questions-6}{%
\subsubsection{Open Questions}\label{open-questions-6}}

\hypertarget{equivalencetest}{%
\section{Equivalence Testing and Interval
Hypotheses}\label{equivalencetest}}

Placeholder

\hypertarget{equivalence-tests}{%
\subsection{Equivalence tests}\label{equivalence-tests}}

\hypertarget{reporting-equivalence-tests}{%
\subsection{Reporting Equivalence
Tests}\label{reporting-equivalence-tests}}

\hypertarget{MET}{%
\subsection{Minimum Effect Tests}\label{MET}}

\hypertarget{power-analysis-for-interval-hypothesis-tests}{%
\subsection{Power Analysis for Interval Hypothesis
Tests}\label{power-analysis-for-interval-hypothesis-tests}}

\hypertarget{ROPE}{%
\subsection{The Bayesian ROPE procedure}\label{ROPE}}

\hypertarget{whichinterval}{%
\subsection{Which interval width should be used?}\label{whichinterval}}

\hypertarget{sesoi}{%
\subsection{Setting the Smallest Effect Size of Interest}\label{sesoi}}

\hypertarget{specifying-a-sesoi-based-on-theory}{%
\subsection{Specifying a SESOI based on
theory}\label{specifying-a-sesoi-based-on-theory}}

\hypertarget{anchor-based-methods-to-set-a-sesoi}{%
\subsection{Anchor based methods to set a
SESOI}\label{anchor-based-methods-to-set-a-sesoi}}

\hypertarget{specifying-a-sesoi-based-on-a-cost-benefit-analysis}{%
\subsection{Specifying a SESOI based on a cost-benefit
analysis}\label{specifying-a-sesoi-based-on-a-cost-benefit-analysis}}

\hypertarget{specifying-the-sesoi-using-the-small-telescopes-approach}{%
\subsection{Specifying the SESOI using the small telescopes
approach}\label{specifying-the-sesoi-using-the-small-telescopes-approach}}

\hypertarget{setting-the-smallest-effect-size-of-interest-to-the-minimal-statistically-detectable-effect}{%
\subsection{Setting the Smallest Effect Size of Interest to the Minimal
Statistically Detectable
Effect}\label{setting-the-smallest-effect-size-of-interest-to-the-minimal-statistically-detectable-effect}}

\hypertarget{test-yourself-7}{%
\subsection{Test Yourself}\label{test-yourself-7}}

\hypertarget{questions-about-equivalence-tests}{%
\subsubsection{Questions about equivalence
tests}\label{questions-about-equivalence-tests}}

\hypertarget{questions-about-the-small-telescopes-approach}{%
\subsubsection{Questions about the small telescopes
approach}\label{questions-about-the-small-telescopes-approach}}

\hypertarget{questions-about-specifying-the-sesoi-as-the-minimal-statistically-detectable-effect}{%
\subsubsection{Questions about specifying the SESOI as the Minimal
Statistically Detectable
Effect}\label{questions-about-specifying-the-sesoi-as-the-minimal-statistically-detectable-effect}}

\hypertarget{open-questions-7}{%
\subsubsection{Open Questions}\label{open-questions-7}}

\hypertarget{sequential}{%
\section{Sequential Analysis}\label{sequential}}

Placeholder

\hypertarget{choosing-alpha-levels-for-sequential-analyses.}{%
\subsection{Choosing alpha levels for sequential
analyses.}\label{choosing-alpha-levels-for-sequential-analyses.}}

\hypertarget{the-pocock-correction}{%
\subsection{The Pocock correction}\label{the-pocock-correction}}

\hypertarget{comparing-spending-functions}{%
\subsection{Comparing Spending
Functions}\label{comparing-spending-functions}}

\hypertarget{alpha-spending-functions}{%
\subsection{Alpha spending functions}\label{alpha-spending-functions}}

\hypertarget{updating-boundaries-during-a-study}{%
\subsection{Updating boundaries during a
study}\label{updating-boundaries-during-a-study}}

\hypertarget{sample-size-for-sequential-designs}{%
\subsection{Sample Size for Sequential
Designs}\label{sample-size-for-sequential-designs}}

\hypertarget{stopping-for-futility}{%
\subsection{Stopping for futility}\label{stopping-for-futility}}

\hypertarget{reporting-the-results-of-a-sequential-analysis}{%
\subsection{Reporting the results of a sequential
analysis}\label{reporting-the-results-of-a-sequential-analysis}}

\hypertarget{test-yourself-8}{%
\subsection{Test Yourself}\label{test-yourself-8}}

\hypertarget{open-questions-8}{%
\subsubsection{Open Questions}\label{open-questions-8}}

\hypertarget{meta}{%
\section{Meta-analysis}\label{meta}}

Placeholder

\hypertarget{random-variation}{%
\subsection{Random Variation}\label{random-variation}}

\hypertarget{a-single-study-meta-analysis}{%
\subsection{A single study
meta-analysis}\label{a-single-study-meta-analysis}}

\hypertarget{simulating-meta-analyses-of-mean-standardized-differences}{%
\subsection{Simulating meta-analyses of mean standardized
differences}\label{simulating-meta-analyses-of-mean-standardized-differences}}

\hypertarget{fixed-effect-vs-random-effects}{%
\subsection{Fixed Effect vs Random
Effects}\label{fixed-effect-vs-random-effects}}

\hypertarget{simulating-meta-analyses-for-dichotomous-outcomes}{%
\subsection{Simulating meta-analyses for dichotomous
outcomes}\label{simulating-meta-analyses-for-dichotomous-outcomes}}

\hypertarget{heterogeneity}{%
\subsection{Heterogeneity}\label{heterogeneity}}

\hypertarget{strengths-and-weaknesses-of-meta-analysis}{%
\subsection{Strengths and weaknesses of
meta-analysis}\label{strengths-and-weaknesses-of-meta-analysis}}

\hypertarget{which-results-should-you-report-to-be-included-in-a-future-meta-analysis}{%
\subsection{Which results should you report to be included in a future
meta-analysis?}\label{which-results-should-you-report-to-be-included-in-a-future-meta-analysis}}

\hypertarget{metareporting}{%
\subsection{Improving the reproducibility of
meta-analyses}\label{metareporting}}

\hypertarget{test-yourself-9}{%
\subsection{Test Yourself}\label{test-yourself-9}}

\hypertarget{open-questions-9}{%
\subsubsection{Open Questions}\label{open-questions-9}}

\hypertarget{bias}{%
\section{Bias detection}\label{bias}}

Placeholder

\hypertarget{publicationbias}{%
\subsection{Publication bias}\label{publicationbias}}

\hypertarget{bias-detection-in-meta-analysis}{%
\subsection{Bias detection in
meta-analysis}\label{bias-detection-in-meta-analysis}}

\hypertarget{trim-and-fill}{%
\subsection{Trim and Fill}\label{trim-and-fill}}

\hypertarget{pet-peese}{%
\subsection{PET-PEESE}\label{pet-peese}}

\hypertarget{p-value-meta-analysis}{%
\subsection{\texorpdfstring{\emph{P}-value
meta-analysis}{P-value meta-analysis}}\label{p-value-meta-analysis}}

\hypertarget{conclusion}{%
\subsection{Conclusion}\label{conclusion}}

\hypertarget{test-yourself-10}{%
\subsection{Test Yourself}\label{test-yourself-10}}

\hypertarget{open-questions-10}{%
\subsubsection{Open Questions}\label{open-questions-10}}

\hypertarget{prereg}{%
\section{Preregistration and Transparency}\label{prereg}}

Placeholder

\hypertarget{preregistration-of-the-statistical-analysis-plan}{%
\subsection{Preregistration of the Statistical Analysis
Plan}\label{preregistration-of-the-statistical-analysis-plan}}

\hypertarget{the-value-of-preregistration}{%
\subsection{The value of
preregistration}\label{the-value-of-preregistration}}

\hypertarget{how-to-preregister}{%
\subsection{How to preregister}\label{how-to-preregister}}

\hypertarget{journal-article-reporting-standards}{%
\subsection{Journal Article Reporting
Standards}\label{journal-article-reporting-standards}}

\hypertarget{what-does-a-formalized-test-of-a-prediction-look-like}{%
\subsection{What Does a Formalized Test of a Prediction Look
Like?}\label{what-does-a-formalized-test-of-a-prediction-look-like}}

\hypertarget{are-you-ready-to-preregister-a-hypothesis-test}{%
\subsection{Are you ready to preregister a hypothesis
test?}\label{are-you-ready-to-preregister-a-hypothesis-test}}

\hypertarget{test-yourself-11}{%
\subsection{Test Yourself}\label{test-yourself-11}}

\hypertarget{pre-registering-on-aspredicted}{%
\subsection{Pre-registering on
AsPredicted}\label{pre-registering-on-aspredicted}}

\hypertarget{pre-registering-on-the-open-science-framework}{%
\subsection{Pre-registering on the Open Science
Framework}\label{pre-registering-on-the-open-science-framework}}

\hypertarget{collecting-data}{%
\subsection{Collecting Data}\label{collecting-data}}

\hypertarget{analyzing-the-data}{%
\subsection{Analyzing the data}\label{analyzing-the-data}}

\hypertarget{sharing-the-report-data-and-code}{%
\subsection{Sharing the report, data, and
code}\label{sharing-the-report-data-and-code}}

\hypertarget{computationalreproducibility}{%
\section{Computational
Reproducibility}\label{computationalreproducibility}}

Placeholder

\hypertarget{step-1-setting-up-a-github-repository}{%
\subsection{Step 1: Setting up a GitHub
repository}\label{step-1-setting-up-a-github-repository}}

\hypertarget{step-2-cloning-your-github-repository-into-rstudio}{%
\subsection{Step 2: Cloning your GitHub repository into
RStudio}\label{step-2-cloning-your-github-repository-into-rstudio}}

\hypertarget{step-3-creating-an-r-markdown-file}{%
\subsection{Step 3: Creating an R Markdown
file}\label{step-3-creating-an-r-markdown-file}}

\hypertarget{step-4-reproducible-data-analysis-in-r-studio}{%
\subsection{Step 4: Reproducible Data Analysis in R
Studio}\label{step-4-reproducible-data-analysis-in-r-studio}}

\hypertarget{step-5-committing-and-pushing-to-github}{%
\subsection{Step 5: Committing and Pushing to
GitHub}\label{step-5-committing-and-pushing-to-github}}

\hypertarget{step-6-reproducible-data-analysis}{%
\subsection{Step 6: Reproducible Data
Analysis}\label{step-6-reproducible-data-analysis}}

\hypertarget{extra-apa-formatted-manuscripts-in-papaja}{%
\subsubsection{Extra: APA formatted manuscripts in
papaja}\label{extra-apa-formatted-manuscripts-in-papaja}}

\hypertarget{step-7-organizing-your-data-and-code}{%
\subsection{Step 7: Organizing Your Data and
Code}\label{step-7-organizing-your-data-and-code}}

\hypertarget{step-8-archiving-your-data-and-code}{%
\subsection{Step 8: Archiving Your Data and
Code}\label{step-8-archiving-your-data-and-code}}

\hypertarget{extra-sharing-reproducible-code-on-code-ocean}{%
\subsubsection{EXTRA: Sharing Reproducible Code on Code
Ocean}\label{extra-sharing-reproducible-code-on-code-ocean}}

\hypertarget{some-points-for-improvement-in-computational-reproducibility}{%
\subsection{Some points for improvement in computational
reproducibility}\label{some-points-for-improvement-in-computational-reproducibility}}

\hypertarget{conclusion-1}{%
\subsection{Conclusion}\label{conclusion-1}}

\hypertarget{refs}{}
\begin{CSLReferences}{1}{0}
\leavevmode\vadjust pre{\hypertarget{ref-abelson_value_2003}{}}%
Abelson, P. (2003). The {Value} of {Life} and {Health} for {Public
Policy}. \emph{Economic Record}, \emph{79}, S2--S13.
\url{https://doi.org/10.1111/1475-4932.00087}

\leavevmode\vadjust pre{\hypertarget{ref-aberson_applied_2019}{}}%
Aberson, C. L. (2019). \emph{Applied {Power Analysis} for the
{Behavioral Sciences}} (Second). {Routledge}.

\leavevmode\vadjust pre{\hypertarget{ref-aert_correcting_2018}{}}%
Aert, R. C. M. van, \& Assen, M. A. L. M. van. (2018). \emph{Correcting
for {Publication Bias} in a {Meta-Analysis} with the {P-uniform}*
{Method}}. {MetaArXiv}. \url{https://doi.org/10.31222/osf.io/zqjr9}

\leavevmode\vadjust pre{\hypertarget{ref-albers_credible_2018}{}}%
Albers, C. J., Kiers, H. A. L., \& Ravenzwaaij, D. van. (2018). Credible
{Confidence}: {A Pragmatic View} on the {Frequentist} vs {Bayesian
Debate}. \emph{Collabra: Psychology}, \emph{4}(1), 31.
\url{https://doi.org/10.1525/collabra.149}

\leavevmode\vadjust pre{\hypertarget{ref-albers_when_2018}{}}%
Albers, C. J., \& Lakens, D. (2018). When power analyses based on pilot
data are biased: {Inaccurate} effect size estimators and follow-up bias.
\emph{Journal of Experimental Social Psychology}, \emph{74}, 187--195.
\url{https://doi.org/10.1016/j.jesp.2017.09.004}

\leavevmode\vadjust pre{\hypertarget{ref-aldrich_r_1997}{}}%
Aldrich, J. (1997). R.{A}. {Fisher} and the making of maximum likelihood
1912-1922. \emph{Statistical Science}, \emph{12}(3), 162--176.
\url{https://doi.org/10.1214/ss/1030037906}

\leavevmode\vadjust pre{\hypertarget{ref-allison_power_1997}{}}%
Allison, D. B., Allison, R. L., Faith, M. S., Paultre, F., \& Pi-Sunyer,
F. X. (1997). Power and money: {Designing} statistically powerful
studies while minimizing financial costs. \emph{Psychological Methods},
\emph{2}(1), 20--33. \url{https://doi.org/10.1037/1082-989X.2.1.20}

\leavevmode\vadjust pre{\hypertarget{ref-altman_statistics_1995}{}}%
Altman, D. G., \& Bland, J. M. (1995). Statistics notes: {Absence} of
evidence is not evidence of absence. \emph{BMJ}, \emph{311}(7003), 485.
\url{https://doi.org/10.1136/bmj.311.7003.485}

\leavevmode\vadjust pre{\hypertarget{ref-anderson_group_2014}{}}%
Anderson, K. M. (2014). Group {Sequential Design} in {R}. In
\emph{Clinical {Trial Biostatistics} and {Biopharmaceutical
Applications}} (pp. 179--209). {CRC Press}.

\leavevmode\vadjust pre{\hypertarget{ref-anderson_perverse_2007}{}}%
Anderson, M. S., Ronning, E. A., De Vries, R., \& Martinson, B. C.
(2007). The perverse effects of competition on scientists' work and
relationships. \emph{Science and Engineering Ethics}, \emph{13}(4),
437--461.

\leavevmode\vadjust pre{\hypertarget{ref-R-BUCSS}{}}%
Anderson, S. F., \& Kelley, K. (2020). \emph{BUCSS: Bias and uncertainty
corrected sample size}. \url{https://CRAN.R-project.org/package=BUCSS}

\leavevmode\vadjust pre{\hypertarget{ref-anderson_sample-size_2017}{}}%
Anderson, S. F., Kelley, K., \& Maxwell, S. E. (2017). Sample-size
planning for more accurate statistical power: {A} method adjusting
sample effect sizes for publication bias and uncertainty.
\emph{Psychological Science}, \emph{28}(11), 1547--1562.
\url{https://doi.org/10.1177/0956797617723724}

\leavevmode\vadjust pre{\hypertarget{ref-anderson_addressing_2017}{}}%
Anderson, S. F., \& Maxwell, S. E. (2017). Addressing the
{``{Replication Crisis}''}: {Using Original Studies} to {Design
Replication Studies} with {Appropriate Statistical Power}.
\emph{Multivariate Behavioral Research}, 1--20.
\url{https://doi.org/10.1080/00273171.2017.1289361}

\leavevmode\vadjust pre{\hypertarget{ref-anderson_theres_2016}{}}%
Anderson, S. F., \& Maxwell, S. E. (2016). There's more than one way to
conduct a replication study: {Beyond} statistical significance.
\emph{Psychological Methods}, \emph{21}(1), 1--12.
\url{https://doi.org/10.1037/met0000051}

\leavevmode\vadjust pre{\hypertarget{ref-anvari_not_2021}{}}%
Anvari, F., Kievit, R., Lakens, D., Pennington, C. R., Przybylski, A.
K., Tiokhin, L., Wiernik, B. M., \& Orben, A. (2021). Not all effects
are indispensable: {Psychological} science requires verifiable lines of
reasoning for whether an effect matters. \emph{Perspectives on
Psychological Science}. \url{https://doi.org/10.31234/osf.io/g3vtr}

\leavevmode\vadjust pre{\hypertarget{ref-anvari_using_2021}{}}%
Anvari, F., \& Lakens, D. (2021). Using anchor-based methods to
determine the smallest effect size of interest. \emph{Journal of
Experimental Social Psychology}, \emph{96}, 104159.
\url{https://doi.org/10.1016/j.jesp.2021.104159}

\leavevmode\vadjust pre{\hypertarget{ref-appelbaum_journal_2018}{}}%
Appelbaum, M., Cooper, H., Kline, R. B., Mayo-Wilson, E., Nezu, A. M.,
\& Rao, S. M. (2018). Journal article reporting standards for
quantitative research in psychology: {The APA Publications} and
{Communications Board} task force report. \emph{American Psychologist},
\emph{73}(1), 3. \url{https://doi.org/10.1037/amp0000191}

\leavevmode\vadjust pre{\hypertarget{ref-armitage_repeated_1969}{}}%
Armitage, P., McPherson, C. K., \& Rowe, B. C. (1969). Repeated
significance tests on accumulating data. \emph{Journal of the Royal
Statistical Society: Series A (General)}, \emph{132}(2), 235--244.

\leavevmode\vadjust pre{\hypertarget{ref-arslan_how_2019}{}}%
Arslan, R. C. (2019). How to {Automatically Document Data With} the
codebook {Package} to {Facilitate Data Reuse}. \emph{Advances in Methods
and Practices in Psychological Science}, 2515245919838783.
\url{https://doi.org/10.1177/2515245919838783}

\leavevmode\vadjust pre{\hypertarget{ref-R-gridExtra}{}}%
Auguie, B. (2017). \emph{gridExtra: Miscellaneous functions for "grid"
graphics}. \url{https://CRAN.R-project.org/package=gridExtra}

\leavevmode\vadjust pre{\hypertarget{ref-babbage_reflections_1830}{}}%
Babbage, C. (1830). \emph{Reflections on the {Decline} of {Science} in
{England}: {And} on {Some} of {Its Causes}}. {B. Fellowes}.

\leavevmode\vadjust pre{\hypertarget{ref-bacchetti_current_2010}{}}%
Bacchetti, P. (2010). Current sample size conventions: {Flaws}, harms,
and alternatives. \emph{BMC Medicine}, \emph{8}(1), 17.
\url{https://doi.org/10.1186/1741-7015-8-17}

\leavevmode\vadjust pre{\hypertarget{ref-baguley_understanding_2004}{}}%
Baguley, T. (2004). Understanding statistical power in the context of
applied research. \emph{Applied Ergonomics}, \emph{35}(2), 73--80.
\url{https://doi.org/10.1016/j.apergo.2004.01.002}

\leavevmode\vadjust pre{\hypertarget{ref-baguley_standardized_2009}{}}%
Baguley, T. (2009). Standardized or simple effect size: {What} should be
reported? \emph{British Journal of Psychology}, \emph{100}(3), 603--617.
\url{https://doi.org/10.1348/000712608X377117}

\leavevmode\vadjust pre{\hypertarget{ref-baguley_serious_2012}{}}%
Baguley, T. (2012). \emph{Serious stats: A guide to advanced statistics
for the behavioral sciences}. {Palgrave Macmillan}.

\leavevmode\vadjust pre{\hypertarget{ref-bakan_test_1966}{}}%
Bakan, D. (1966). The test of significance in psychological research.
\emph{Psychological Bulletin}, \emph{66}(6), 423--437.
\url{https://doi.org/10.1037/h0020412}

\leavevmode\vadjust pre{\hypertarget{ref-ball_effects_2002}{}}%
Ball, K., Berch, D. B., Helmers, K. F., Jobe, J. B., Leveck, M. D.,
Marsiske, M., Morris, J. N., Rebok, G. W., Smith, D. M., \& Tennstedt,
S. L. (2002). Effects of cognitive training interventions with older
adults: A randomized controlled trial. \emph{Jama}, \emph{288}(18),
2271--2281.

\leavevmode\vadjust pre{\hypertarget{ref-barber_pitfalls_1976}{}}%
Barber, T. X. (1976). \emph{Pitfalls in {Human Research}: {Ten Pivotal
Points}}. {Pergamon Press}.

\leavevmode\vadjust pre{\hypertarget{ref-bartos_z-curve20_2020}{}}%
Bartoš, F., \& Schimmack, U. (2020). \emph{Z-{Curve}.2.0: {Estimating
Replication Rates} and {Discovery Rates}}.
\url{https://doi.org/10.31234/osf.io/urgtn}

\leavevmode\vadjust pre{\hypertarget{ref-R-zcurve}{}}%
Bartoš, F., \& Schimmack, U. (2022). \emph{Zcurve: An implementation of
z-curves}. \url{https://fbartos.github.io/zcurve/}

\leavevmode\vadjust pre{\hypertarget{ref-R-Matrix}{}}%
Bates, D., \& Maechler, M. (2021). \emph{Matrix: Sparse and dense matrix
classes and methods}. \url{http://Matrix.R-forge.R-project.org/}

\leavevmode\vadjust pre{\hypertarget{ref-bauer_unifying_1996}{}}%
Bauer, P., \& Kieser, M. (1996). A unifying approach for confidence
intervals and testing of equivalence and difference. \emph{Biometrika},
\emph{83}(4), 934--937.

\leavevmode\vadjust pre{\hypertarget{ref-bausell_power_2002}{}}%
Bausell, R. B., \& Li, Y.-F. (2002). \emph{Power {Analysis} for
{Experimental Research}: {A Practical Guide} for the {Biological},
{Medical} and {Social Sciences}} (1st edition). {Cambridge University
Press}.

\leavevmode\vadjust pre{\hypertarget{ref-becker_failsafe_2005}{}}%
Becker, B. J. (2005). Failsafe {N} or {File-Drawer Number}. In
\emph{Publication {Bias} in {Meta-Analysis}} (pp. 111--125). {John Wiley
\& Sons, Ltd}. \url{https://doi.org/10.1002/0470870168.ch7}

\leavevmode\vadjust pre{\hypertarget{ref-bem_feeling_2011}{}}%
Bem, D. J. (2011). Feeling the future: Experimental evidence for
anomalous retroactive influences on cognition and affect. \emph{Journal
of Personality and Social Psychology}, \emph{100}(3), 407--425.
\url{https://doi.org/10.1037/a0021524}

\leavevmode\vadjust pre{\hypertarget{ref-bender_adjusting_2001}{}}%
Bender, R., \& Lange, S. (2001). Adjusting for multiple
testing\textemdash when and how? \emph{Journal of Clinical
Epidemiology}, \emph{54}(4), 343--349.

\leavevmode\vadjust pre{\hypertarget{ref-benjamini_its_2016}{}}%
Benjamini, Y. (2016). It's {Not} the p-values' {Fault}. \emph{The
American Statistician: Supplemental Material to the ASA Statement on
P-Values and Statistical Significance}, \emph{70}, 1--2.

\leavevmode\vadjust pre{\hypertarget{ref-benjamini_controlling_1995}{}}%
Benjamini, Y., \& Hochberg, Y. (1995). Controlling the false discovery
rate: A practical and powerful approach to multiple testing.
\emph{Journal of the Royal Statistical Society. Series B
(Methodological)}, 289--300.

\leavevmode\vadjust pre{\hypertarget{ref-ben-shachar_effectsize_2020}{}}%
Ben-Shachar, M. S., Lüdecke, D., \& Makowski, D. (2020a). Effectsize:
{Estimation} of {Effect Size Indices} and {Standardized Parameters}.
\emph{Journal of Open Source Software}, \emph{5}(56), 2815.
\url{https://doi.org/10.21105/joss.02815}

\leavevmode\vadjust pre{\hypertarget{ref-effectsize2020}{}}%
Ben-Shachar, M. S., Lüdecke, D., \& Makowski, D. (2020b). {e}ffectsize:
Estimation of effect size indices and standardized parameters.
\emph{Journal of Open Source Software}, \emph{5}(56), 2815.
\url{https://doi.org/10.21105/joss.02815}

\leavevmode\vadjust pre{\hypertarget{ref-berger_interplay_2004}{}}%
Berger, J. O., \& Bayarri, M. J. (2004). The {Interplay} of {Bayesian}
and {Frequentist Analysis}. \emph{Statistical Science}, \emph{19}(1),
58--80. \url{https://doi.org/10.1214/088342304000000116}

\leavevmode\vadjust pre{\hypertarget{ref-berkeley_defence_1735}{}}%
Berkeley, G. (1735). \emph{A defence of free-thinking in mathematics, in
answer to a pamphlet of {Philalethes Cantabrigiensis} entitled {Geometry
No Friend} to {Infidelity}. {Also} an appendix concerning mr. {Walton}'s
{Vindication} of the principles of fluxions against the objections
contained in {The} analyst. {By} the author of {The} minute philosopher}
(Vol. 3).

\leavevmode\vadjust pre{\hypertarget{ref-bishop_fallibility_2018}{}}%
Bishop, D. V. M. (2018). Fallibility in {Science}: {Responding} to
{Errors} in the {Work} of {Oneself} and {Others}. \emph{Advances in
Methods and Practices in Psychological Science}, 2515245918776632.
\url{https://doi.org/10.1177/2515245918776632}

\leavevmode\vadjust pre{\hypertarget{ref-bland_introduction_2015}{}}%
Bland, M. (2015). \emph{An introduction to medical statistics} (Fourth
edition). {Oxford University Press}.

\leavevmode\vadjust pre{\hypertarget{ref-borenstein_introduction_2009}{}}%
Borenstein, M. (Ed.). (2009). \emph{Introduction to meta-analysis}.
{John Wiley \& Sons}.

\leavevmode\vadjust pre{\hypertarget{ref-bosco_correlational_2015}{}}%
Bosco, F. A., Aguinis, H., Singh, K., Field, J. G., \& Pierce, C. A.
(2015). Correlational effect size benchmarks. \emph{The Journal of
Applied Psychology}, \emph{100}(2), 431--449.
\url{https://doi.org/10.1037/a0038047}

\leavevmode\vadjust pre{\hypertarget{ref-bretz_multiple_2011}{}}%
Bretz, F., Hothorn, T., \& Westfall, P. H. (2011). \emph{Multiple
comparisons using {R}}. {CRC Press}.

\leavevmode\vadjust pre{\hypertarget{ref-bross_critical_1971}{}}%
Bross, I. D. (1971). Critical levels, statistical language and
scientific inference. In \emph{Foundations of statistical inference}
(pp. 500--513). {Holt, Rinehart and Winston}.

\leavevmode\vadjust pre{\hypertarget{ref-brown_errors_1983}{}}%
Brown, G. W. (1983). Errors, {Types I} and {II}. \emph{American Journal
of Diseases of Children}, \emph{137}(6), 586--591.
\url{https://doi.org/10.1001/archpedi.1983.02140320062014}

\leavevmode\vadjust pre{\hypertarget{ref-brown_grim_2017}{}}%
Brown, N. J. L., \& Heathers, J. A. J. (2017). The {GRIM Test}: {A
Simple Technique Detects Numerous Anomalies} in the {Reporting} of
{Results} in {Psychology}. \emph{Social Psychological and Personality
Science}, \emph{8}(4), 363--369.
\url{https://doi.org/10.1177/1948550616673876}

\leavevmode\vadjust pre{\hypertarget{ref-brunner_estimating_2020}{}}%
Brunner, J., \& Schimmack, U. (2020). Estimating {Population Mean Power
Under Conditions} of {Heterogeneity} and {Selection} for {Significance}.
\emph{Meta-Psychology}, \emph{4}.
\url{https://doi.org/10.15626/MP.2018.874}

\leavevmode\vadjust pre{\hypertarget{ref-bryan_behavioural_2021}{}}%
Bryan, C. J., Tipton, E., \& Yeager, D. S. (2021). Behavioural science
is unlikely to change the world without a heterogeneity revolution.
\emph{Nature Human Behaviour}, 1--10.
\url{https://doi.org/10.1038/s41562-021-01143-3}

\leavevmode\vadjust pre{\hypertarget{ref-brysbaert_how_2019}{}}%
Brysbaert, M. (2019). How many participants do we have to include in
properly powered experiments? {A} tutorial of power analysis with
reference tables. \emph{Journal of Cognition}, \emph{2}(1), 16.
\url{https://doi.org/10.5334/joc.72}

\leavevmode\vadjust pre{\hypertarget{ref-brysbaert_power_2018}{}}%
Brysbaert, M., \& Stevens, M. (2018). Power {Analysis} and {Effect Size}
in {Mixed Effects Models}: {A Tutorial}. \emph{Journal of Cognition},
\emph{1}(1). \url{https://doi.org/10.5334/joc.10}

\leavevmode\vadjust pre{\hypertarget{ref-R-MOTE}{}}%
Buchanan, E. M., Gillenwaters, A. M., Scofield, J. E., \& Valentine, K.
D. (2019). \emph{MOTE: Effect size and confidence interval calculator}.
\url{https://CRAN.R-project.org/package=MOTE}

\leavevmode\vadjust pre{\hypertarget{ref-buchanan_mote_2017}{}}%
Buchanan, E. M., Scofield, J., \& Valentine, K. D. (2017). \emph{{MOTE}:
{Effect Size} and {Confidence Interval Calculator}.}

\leavevmode\vadjust pre{\hypertarget{ref-bulus_bound_2021}{}}%
Bulus, M., \& Dong, N. (2021). Bound {Constrained Optimization} of
{Sample Sizes Subject} to {Monetary Restrictions} in {Planning
Multilevel Randomized Trials} and {Regression Discontinuity Studies}.
\emph{The Journal of Experimental Education}, \emph{89}(2), 379--401.
\url{https://doi.org/10.1080/00220973.2019.1636197}

\leavevmode\vadjust pre{\hypertarget{ref-burriss_changes_2015}{}}%
Burriss, R. P., Troscianko, J., Lovell, P. G., Fulford, A. J. C.,
Stevens, M., Quigley, R., Payne, J., Saxton, T. K., \& Rowland, H. M.
(2015). Changes in women's facial skin color over the ovulatory cycle
are not detectable by the human visual system. \emph{PLOS ONE},
\emph{10}(7), e0130093.
\url{https://doi.org/10.1371/journal.pone.0130093}

\leavevmode\vadjust pre{\hypertarget{ref-button_power_2013}{}}%
Button, K. S., Ioannidis, J. P. A., Mokrysz, C., Nosek, B. A., Flint,
J., Robinson, E. S. J., \& Munafò, M. R. (2013). Power failure: Why
small sample size undermines the reliability of neuroscience.
\emph{Nature Reviews Neuroscience}, \emph{14}(5), 365--376.
\url{https://doi.org/10.1038/nrn3475}

\leavevmode\vadjust pre{\hypertarget{ref-button_minimal_2015}{}}%
Button, K. S., Kounali, D., Thomas, L., Wiles, N. J., Peters, T. J.,
Welton, N. J., Ades, A. E., \& Lewis, G. (2015). Minimal clinically
important difference on the {Beck Depression Inventory} - {II} according
to the patient's perspective. \emph{Psychological Medicine},
\emph{45}(15), 3269--3279.
\url{https://doi.org/10.1017/S0033291715001270}

\leavevmode\vadjust pre{\hypertarget{ref-R-Superpower}{}}%
Caldwell, A., \& Lakens, D. (2021). \emph{Superpower: Simulation-based
power analysis for factorial designs}.
\url{https://aaroncaldwell.us/SuperpowerBook/}

\leavevmode\vadjust pre{\hypertarget{ref-carter_publication_2014}{}}%
Carter, E. C., \& McCullough, M. E. (2014). Publication bias and the
limited strength model of self-control: Has the evidence for ego
depletion been overestimated? \emph{Frontiers in Psychology}, \emph{5}.
\url{https://doi.org/10.3389/fpsyg.2014.00823}

\leavevmode\vadjust pre{\hypertarget{ref-carter_correcting_2019}{}}%
Carter, E. C., Schönbrodt, F. D., Gervais, W. M., \& Hilgard, J. (2019).
Correcting for {Bias} in {Psychology}: {A Comparison} of {Meta-Analytic
Methods}. \emph{Advances in Methods and Practices in Psychological
Science}, \emph{2}(2), 115--144.
\url{https://doi.org/10.1177/2515245919847196}

\leavevmode\vadjust pre{\hypertarget{ref-cascio_open_1983}{}}%
Cascio, W. F., \& Zedeck, S. (1983). Open a {New Window} in {Rational
Research Planning}: {Adjust Alpha} to {Maximize Statistical Power}.
\emph{Personnel Psychology}, \emph{36}(3), 517--526.
\url{https://doi.org/10.1111/j.1744-6570.1983.tb02233.x}

\leavevmode\vadjust pre{\hypertarget{ref-chambers_past_2022}{}}%
Chambers, C. D., \& Tzavella, L. (2022). The past, present and future of
{Registered Reports}. \emph{Nature Human Behaviour}, \emph{6}(1),
29--42. \url{https://doi.org/10.1038/s41562-021-01193-7}

\leavevmode\vadjust pre{\hypertarget{ref-R-pwr}{}}%
Champely, S. (2020). \emph{Pwr: Basic functions for power analysis}.
\url{https://github.com/heliosdrm/pwr}

\leavevmode\vadjust pre{\hypertarget{ref-chang_adaptive_2016}{}}%
Chang, M. (2016). \emph{Adaptive {Design Theory} and {Implementation
Using SAS} and {R}} (2nd edition). {Chapman and Hall/CRC}.

\leavevmode\vadjust pre{\hypertarget{ref-chatziathanasiou_beware_2022}{}}%
Chatziathanasiou, K. (2022). \emph{Beware the {Lure} of {Narratives}:
{``{Hungry Judges}''} {Should} not {Motivate} the {Use} of
{``{Artificial Intelligence}''} in {Law}} (\{\{SSRN Scholarly Paper\}\}
ID 4011603). {Social Science Research Network}.
\url{https://doi.org/10.2139/ssrn.4011603}

\leavevmode\vadjust pre{\hypertarget{ref-chin_questionable_2021}{}}%
Chin, J. M., Pickett, J. T., Vazire, S., \& Holcombe, A. O. (2021).
Questionable {Research Practices} and {Open Science} in {Quantitative
Criminology}. \emph{Journal of Quantitative Criminology}.
\url{https://doi.org/10.1007/s10940-021-09525-6}

\leavevmode\vadjust pre{\hypertarget{ref-cho_is_2013}{}}%
Cho, H.-C., \& Abe, S. (2013). Is two-tailed testing for directional
research hypotheses tests legitimate? \emph{Journal of Business
Research}, \emph{66}(9), 1261--1266.
\url{https://doi.org/10.1016/j.jbusres.2012.02.023}

\leavevmode\vadjust pre{\hypertarget{ref-cohen_statistical_1988}{}}%
Cohen, J. (1988). \emph{Statistical power analysis for the behavioral
sciences} (2nd ed). {L. Erlbaum Associates}.

\leavevmode\vadjust pre{\hypertarget{ref-cohen_things_1990}{}}%
Cohen, J. (1990). Things {I} have learned (so far). \emph{American
Psychologist}, \emph{45}(12), 1304--1312.
\url{https://doi.org/10.1037/0003-066X.45.12.1304}

\leavevmode\vadjust pre{\hypertarget{ref-cohen_earth_1994}{}}%
Cohen, J. (1994). The earth is round (p {\(<\)} .05). \emph{American
Psychologist}, \emph{49}(12), 997--1003.
\url{https://doi.org/10.1037/0003-066X.49.12.997}

\leavevmode\vadjust pre{\hypertarget{ref-colling_registered_2020}{}}%
Colling, L. J., Szűcs, D., De Marco, D., Cipora, K., Ulrich, R., Nuerk,
H.-C., Soltanlou, M., Bryce, D., Chen, S.-C., Schroeder, P. A., Henare,
D. T., Chrystall, C. K., Corballis, P. M., Ansari, D., Goffin, C.,
Sokolowski, H. M., Hancock, P. J. B., Millen, A. E., Langton, S. R. H.,
\ldots{} McShane, B. B. (2020). Registered {Replication Report} on
{Fischer}, {Castel}, {Dodd}, and {Pratt} (2003). \emph{Advances in
Methods and Practices in Psychological Science}, \emph{3}(2), 143--162.
\url{https://doi.org/10.1177/2515245920903079}

\leavevmode\vadjust pre{\hypertarget{ref-colquhoun_false_2019}{}}%
Colquhoun, D. (2019). The {False Positive Risk}: {A Proposal Concerning
What} to {Do About} p-{Values}. \emph{The American Statistician},
\emph{73}(sup1), 192--201.
\url{https://doi.org/10.1080/00031305.2018.1529622}

\leavevmode\vadjust pre{\hypertarget{ref-cook_assessing_2014}{}}%
Cook, J., Hislop, J., Adewuyi, T., Harrild, K., Altman, D., Ramsay, C.,
Fraser, C., Buckley, B., Fayers, P., Harvey, I., Briggs, A., Norrie, J.,
Fergusson, D., Ford, I., \& Vale, L. (2014). Assessing methods to
specify the target difference for a randomised controlled trial: {DELTA}
({Difference ELicitation} in {TriAls}) review. \emph{Health Technology
Assessment}, \emph{18}(28). \url{https://doi.org/10.3310/hta18280}

\leavevmode\vadjust pre{\hypertarget{ref-cook_p-value_2002}{}}%
Cook, T. D. (2002). P-{Value Adjustment} in {Sequential Clinical
Trials}. \emph{Biometrics}, \emph{58}(4), 1005--1011.

\leavevmode\vadjust pre{\hypertarget{ref-cooper_reporting_2020}{}}%
Cooper, H. (2020). \emph{Reporting quantitative research in psychology:
{How} to meet {APA Style Journal Article Reporting Standards} (2nd
ed.).} {American Psychological Association}.
\url{https://doi.org/10.1037/0000178-000}

\leavevmode\vadjust pre{\hypertarget{ref-cooper_handbook_2009}{}}%
Cooper, H. M., Hedges, L. V., \& Valentine, J. C. (Eds.). (2009).
\emph{The handbook of research synthesis and meta-analysis} (2nd ed).
{Russell Sage Foundation}.

\leavevmode\vadjust pre{\hypertarget{ref-copay_understanding_2007}{}}%
Copay, A. G., Subach, B. R., Glassman, S. D., Polly, D. W., \& Schuler,
T. C. (2007). Understanding the minimum clinically important difference:
A review of concepts and methods. \emph{The Spine Journal}, \emph{7}(5),
541--546. \url{https://doi.org/10.1016/j.spinee.2007.01.008}

\leavevmode\vadjust pre{\hypertarget{ref-correll_avoid_2020}{}}%
Correll, J., Mellinger, C., McClelland, G. H., \& Judd, C. M. (2020).
Avoid {Cohen}'s {``{Small},''} {``{Medium},''} and {``{Large}''} for
{Power Analysis}. \emph{Trends in Cognitive Sciences}, \emph{24}(3),
200--207. \url{https://doi.org/10.1016/j.tics.2019.12.009}

\leavevmode\vadjust pre{\hypertarget{ref-cousineau_superb_2019}{}}%
Cousineau, D., \& Chiasson, F. (2019). \emph{Superb: {Computes} standard
error and confidence interval of means under various designs and
sampling schemes} {[}Manual{]}.

\leavevmode\vadjust pre{\hypertarget{ref-cowles_origins_1982}{}}%
Cowles, M., \& Davis, C. (1982). On the origins of the. 05 level of
statistical significance. \emph{American Psychologist}, \emph{37}(5),
553.

\leavevmode\vadjust pre{\hypertarget{ref-cox_problems_1958}{}}%
Cox, D. R. (1958). Some {Problems Connected} with {Statistical
Inference}. \emph{Annals of Mathematical Statistics}, \emph{29}(2),
357--372. \url{https://doi.org/10.1214/aoms/1177706618}

\leavevmode\vadjust pre{\hypertarget{ref-cribbie_recommendations_2004}{}}%
Cribbie, R. A., Gruman, J. A., \& Arpin-Cribbie, C. A. (2004).
Recommendations for applying tests of equivalence. \emph{Journal of
Clinical Psychology}, \emph{60}(1), 1--10.

\leavevmode\vadjust pre{\hypertarget{ref-cumming_new_2014}{}}%
Cumming, G. (2014). The {New Statistics}: {Why} and {How}.
\emph{Psychological Science}, \emph{25}(1), 7--29.
\url{https://doi.org/10.1177/0956797613504966}

\leavevmode\vadjust pre{\hypertarget{ref-cumming_replication_2008}{}}%
Cumming, G. (2008). Replication and {\emph{p}} {Intervals}: {\emph{p}}
{Values Predict} the {Future Only Vaguely}, but {Confidence Intervals Do
Much Better}. \emph{Perspectives on Psychological Science}, \emph{3}(4),
286--300. \url{https://doi.org/10.1111/j.1745-6924.2008.00079.x}

\leavevmode\vadjust pre{\hypertarget{ref-cumming_understanding_2013}{}}%
Cumming, G. (2013). \emph{Understanding the new statistics: {Effect}
sizes, confidence intervals, and meta-analysis}. {Routledge}.

\leavevmode\vadjust pre{\hypertarget{ref-cumming_introduction_2016}{}}%
Cumming, G., \& Calin-Jageman, R. (2016). \emph{Introduction to the {New
Statistics}: {Estimation}, {Open Science}, and {Beyond}}. {Routledge}.

\leavevmode\vadjust pre{\hypertarget{ref-cumming_confidence_2006}{}}%
Cumming, G., \& Maillardet, R. (2006). Confidence intervals and
replication: {Where} will the next mean fall? \emph{Psychological
Methods}, \emph{11}(3), 217--227.
\url{https://doi.org/10.1037/1082-989X.11.3.217}

\leavevmode\vadjust pre{\hypertarget{ref-danziger_extraneous_2011}{}}%
Danziger, S., Levav, J., \& Avnaim-Pesso, L. (2011). Extraneous factors
in judicial decisions. \emph{Proceedings of the National Academy of
Sciences}, \emph{108}(17), 6889--6892.
\url{https://doi.org/10.1073/PNAS.1018033108}

\leavevmode\vadjust pre{\hypertarget{ref-de_groot_methodology_1969}{}}%
de Groot, A. D. (1969). \emph{Methodology} (Vol. 6). {Mouton \& Co.}

\leavevmode\vadjust pre{\hypertarget{ref-debruine_understanding_2021}{}}%
DeBruine, L. M., \& Barr, D. J. (2021). Understanding {Mixed-Effects
Models Through Data Simulation}. \emph{Advances in Methods and Practices
in Psychological Science}, \emph{4}(1), 2515245920965119.
\url{https://doi.org/10.1177/2515245920965119}

\leavevmode\vadjust pre{\hypertarget{ref-delacre_why_2021}{}}%
Delacre, M., Lakens, D., Ley, C., Liu, L., \& Leys, C. (2021). \emph{Why
{Hedges}' g*s based on the non-pooled standard deviation should be
reported with {Welch}'s t-test}. {PsyArXiv}.
\url{https://doi.org/10.31234/osf.io/tu6mp}

\leavevmode\vadjust pre{\hypertarget{ref-delacre_why_2017}{}}%
Delacre, M., Lakens, D., \& Leys, C. (2017). Why {Psychologists Should}
by {Default Use Welch}'s {\emph{t}}-test {Instead} of {Student}'s
{\emph{t}}-test. \emph{International Review of Social Psychology},
\emph{30}(1). \url{https://doi.org/10.5334/irsp.82}

\leavevmode\vadjust pre{\hypertarget{ref-detsky_using_1990}{}}%
Detsky, A. S. (1990). Using cost-effectiveness analysis to improve the
efficiency of allocating funds to clinical trials. \emph{Statistics in
Medicine}, \emph{9}(1-2), 173--184.
\url{https://doi.org/10.1002/sim.4780090124}

\leavevmode\vadjust pre{\hypertarget{ref-dienes_understanding_2008}{}}%
Dienes, Z. (2008). \emph{Understanding psychology as a science: {An}
introduction to scientific and statistical inference}. {Palgrave
Macmillan}.

\leavevmode\vadjust pre{\hypertarget{ref-dienes_using_2014}{}}%
Dienes, Z. (2014). Using {Bayes} to get the most out of non-significant
results. \emph{Frontiers in Psychology}, \emph{5}.
\url{https://doi.org/10.3389/fpsyg.2014.00781}

\leavevmode\vadjust pre{\hypertarget{ref-dmitrienko_traditional_2013}{}}%
Dmitrienko, A., \& D'Agostino Sr, R. (2013). Traditional multiplicity
adjustment methods in clinical trials. \emph{Statistics in Medicine},
\emph{32}(29), 5172--5218. \url{https://doi.org/10.1002/sim.5990}

\leavevmode\vadjust pre{\hypertarget{ref-dodge_method_1929}{}}%
Dodge, H. F., \& Romig, H. G. (1929). A {Method} of {Sampling
Inspection}. \emph{Bell System Technical Journal}, \emph{8}(4),
613--631. \url{https://doi.org/10.1002/j.1538-7305.1929.tb01240.x}

\leavevmode\vadjust pre{\hypertarget{ref-dongen_multiple_2019}{}}%
Dongen, N. N. N. van, Doorn, J. B. van, Gronau, Q. F., Ravenzwaaij, D.
van, Hoekstra, R., Haucke, M. N., Lakens, D., Hennig, C., Morey, R. D.,
Homer, S., Gelman, A., Sprenger, J., \& Wagenmakers, E.-J. (2019).
Multiple {Perspectives} on {Inference} for {Two Simple Statistical
Scenarios}. \emph{The American Statistician}, \emph{73}(sup1), 328--339.
\url{https://doi.org/10.1080/00031305.2019.1565553}

\leavevmode\vadjust pre{\hypertarget{ref-R-binom}{}}%
Dorai-Raj, S. (2014). \emph{Binom: Binomial confidence intervals for
several parameterizations}.
\url{https://CRAN.R-project.org/package=binom}

\leavevmode\vadjust pre{\hypertarget{ref-dubin_theory_1969}{}}%
Dubin, R. (1969). \emph{Theory building}. {Free Press}.

\leavevmode\vadjust pre{\hypertarget{ref-dunn_multiple_1961}{}}%
Dunn, O. J. (1961). Multiple {Comparisons} among {Means}. \emph{Journal
of the American Statistical Association}, \emph{56}(293), 52--64.
\url{https://doi.org/10.1080/01621459.1961.10482090}

\leavevmode\vadjust pre{\hypertarget{ref-dupont_sequential_1983}{}}%
Dupont, W. D. (1983). Sequential stopping rules and sequentially
adjusted {P} values: {Does} one require the other? \emph{Controlled
Clinical Trials}, \emph{4}(1), 3--10.
\url{https://doi.org/10.1016/S0197-2456(83)80003-8}

\leavevmode\vadjust pre{\hypertarget{ref-ebersole_many_2016}{}}%
Ebersole, C. R., Atherton, O. E., Belanger, A. L., Skulborstad, H. M.,
Allen, J. M., Banks, J. B., Baranski, E., Bernstein, M. J., Bonfiglio,
D. B. V., Boucher, L., Brown, E. R., Budiman, N. I., Cairo, A. H.,
Capaldi, C. A., Chartier, C. R., Chung, J. M., Cicero, D. C., Coleman,
J. A., Conway, J. G., \ldots{} Nosek, B. A. (2016). Many {Labs} 3:
{Evaluating} participant pool quality across the academic semester via
replication. \emph{Journal of Experimental Social Psychology},
\emph{67}, 68--82. \url{https://doi.org/10.1016/j.jesp.2015.10.012}

\leavevmode\vadjust pre{\hypertarget{ref-eckermann_value_2010}{}}%
Eckermann, S., Karnon, J., \& Willan, A. R. (2010). The {Value} of
{Value} of {Information}. \emph{PharmacoEconomics}, \emph{28}(9),
699--709. \url{https://doi.org/10.2165/11537370-000000000-00000}

\leavevmode\vadjust pre{\hypertarget{ref-edwards_academic_2017}{}}%
Edwards, M. A., \& Roy, S. (2017). Academic {Research} in the 21st
{Century}: {Maintaining Scientific Integrity} in a {Climate} of
{Perverse Incentives} and {Hypercompetition}. \emph{Environmental
Engineering Science}, \emph{34}(1), 51--61.
\url{https://doi.org/10.1089/ees.2016.0223}

\leavevmode\vadjust pre{\hypertarget{ref-elson_press_2014}{}}%
Elson, M., Mohseni, M. R., Breuer, J., Scharkow, M., \& Quandt, T.
(2014). Press {CRTT} to measure aggressive behavior: The unstandardized
use of the competitive reaction time task in aggression research.
\emph{Psychological Assessment}, \emph{26}(2), 419--432.
\url{https://doi.org/10.1037/a0035569}

\leavevmode\vadjust pre{\hypertarget{ref-erdfelder_gpower_1996}{}}%
Erdfelder, E., Faul, F., \& Buchner, A. (1996). {GPOWER}: {A} general
power analysis program. \emph{Behavior Research Methods, Instruments, \&
Computers}, \emph{28}(1), 1--11.
\url{https://doi.org/10.3758/BF03203630}

\leavevmode\vadjust pre{\hypertarget{ref-eysenck_exercise_1978}{}}%
Eysenck, H. J. (1978). An exercise in mega-silliness. \emph{American
Psychologist}, \emph{33}(5), 517--517.
\url{https://doi.org/10.1037/0003-066X.33.5.517.a}

\leavevmode\vadjust pre{\hypertarget{ref-fanelli_positive_2010}{}}%
Fanelli, D. (2010). {``{Positive}''} {Results Increase Down} the
{Hierarchy} of the {Sciences}. \emph{PLoS ONE}, \emph{5}(4).
\url{https://doi.org/10.1371/journal.pone.0010068}

\leavevmode\vadjust pre{\hypertarget{ref-fanelli_how_2009}{}}%
Fanelli, D. (2009). How {Many Scientists Fabricate} and {Falsify
Research}? {A Systematic Review} and {Meta-Analysis} of {Survey Data}.
\emph{PLOS ONE}, \emph{4}(5), e5738.
\url{https://doi.org/10.1371/journal.pone.0005738}

\leavevmode\vadjust pre{\hypertarget{ref-faul_gpower_2007}{}}%
Faul, F., Erdfelder, E., Lang, A.-G., \& Buchner, A. (2007). {GPower} 3:
{A} flexible statistical power analysis program for the social,
behavioral, and biomedical sciences. \emph{Behavior Research Methods},
\emph{39}(2), 175--191. \url{https://doi.org/10.3758/BF03193146}

\leavevmode\vadjust pre{\hypertarget{ref-ferguson_comment_2014}{}}%
Ferguson, C. J. (2014). Comment: {Why} meta-analyses rarely resolve
ideological debates. \emph{Emotion Review}, \emph{6}(3), 251--252.

\leavevmode\vadjust pre{\hypertarget{ref-ferguson_providing_2021}{}}%
Ferguson, C. J., \& Heene, M. (2021). Providing a lower-bound estimate
for psychology's {``crud factor''}: {The} case of aggression.
\emph{Professional Psychology: Research and Practice}, \emph{52}(6),
620--626. https://doi.org/\url{http://dx.doi.org/10.1037/pro0000386}

\leavevmode\vadjust pre{\hypertarget{ref-ferguson_vast_2012}{}}%
Ferguson, C. J., \& Heene, M. (2012). A vast graveyard of undead
theories publication bias and psychological science's aversion to the
null. \emph{Perspectives on Psychological Science}, \emph{7}(6),
555--561.

\leavevmode\vadjust pre{\hypertarget{ref-ferron_power_1996}{}}%
Ferron, J., \& Onghena, P. (1996). The {Power} of {Randomization Tests}
for {Single-Case Phase Designs}. \emph{The Journal of Experimental
Education}, \emph{64}(3), 231--239.
\url{https://doi.org/10.1080/00220973.1996.9943805}

\leavevmode\vadjust pre{\hypertarget{ref-fiedler_tools_2004}{}}%
Fiedler, K. (2004). Tools, toys, truisms, and theories: {Some} thoughts
on the creative cycle of theory formation. \emph{Personality and Social
Psychology Review}, \emph{8}(2), 123--131.
\url{https://doi.org/10.1207/s15327957pspr0802_5}

\leavevmode\vadjust pre{\hypertarget{ref-fiedler_questionable_2015}{}}%
Fiedler, K., \& Schwarz, N. (2015). Questionable {Research Practices
Revisited}. \emph{Social Psychological and Personality Science},
1948550615612150. \url{https://doi.org/10.1177/1948550615612150}

\leavevmode\vadjust pre{\hypertarget{ref-field_minimizing_2004}{}}%
Field, S. A., Tyre, A. J., Jonzén, N., Rhodes, J. R., \& Possingham, H.
P. (2004). Minimizing the cost of environmental management decisions by
optimizing statistical thresholds. \emph{Ecology Letters}, \emph{7}(8),
669--675. \url{https://doi.org/10.1111/j.1461-0248.2004.00625.x}

\leavevmode\vadjust pre{\hypertarget{ref-fisher_design_1935}{}}%
Fisher, Ronald Aylmer. (1935). \emph{The design of experiments}. {Oliver
And Boyd; Edinburgh; London}.

\leavevmode\vadjust pre{\hypertarget{ref-fisher_statistical_1956}{}}%
Fisher, Ronald A. (1956). \emph{Statistical methods and scientific
inference: Vol. viii}. {Hafner Publishing Co.}

\leavevmode\vadjust pre{\hypertarget{ref-fraley_n-pact_2014}{}}%
Fraley, R. C., \& Vazire, S. (2014). The {N-Pact Factor}: {Evaluating}
the {Quality} of {Empirical Journals} with {Respect} to {Sample Size}
and {Statistical Power}. \emph{PLOS ONE}, \emph{9}(10), e109019.
\url{https://doi.org/10.1371/journal.pone.0109019}

\leavevmode\vadjust pre{\hypertarget{ref-francis_frequency_2014}{}}%
Francis, G. (2014). The frequency of excess success for articles in
{Psychological Science}. \emph{Psychonomic Bulletin \& Review},
\emph{21}(5), 1180--1187.
\url{https://doi.org/10.3758/s13423-014-0601-x}

\leavevmode\vadjust pre{\hypertarget{ref-franco_publication_2014}{}}%
Franco, A., Malhotra, N., \& Simonovits, G. (2014). Publication bias in
the social sciences: {Unlocking} the file drawer. \emph{Science},
\emph{345}(6203), 1502--1505.
\url{https://doi.org/10.1126/SCIENCE.1255484}

\leavevmode\vadjust pre{\hypertarget{ref-freiman_importance_1978}{}}%
Freiman, J. A., Chalmers, T. C., Smith, H., \& Kuebler, R. R. (1978).
The importance of beta, the type {II} error and sample size in the
design and interpretation of the randomized control trial. {Survey} of
71 "negative" trials. \emph{The New England Journal of Medicine},
\emph{299}(13), 690--694.
\url{https://doi.org/10.1056/NEJM197809282991304}

\leavevmode\vadjust pre{\hypertarget{ref-frick_appropriate_1996}{}}%
Frick, R. W. (1996). The appropriate use of null hypothesis testing.
\emph{Psychological Methods}, \emph{1}(4), 379--390.
\url{https://doi.org/10.1037/1082-989X.1.4.379}

\leavevmode\vadjust pre{\hypertarget{ref-fricker_assessing_2019}{}}%
Fricker, R. D., Burke, K., Han, X., \& Woodall, W. H. (2019). Assessing
the {Statistical Analyses Used} in {Basic} and {Applied Social
Psychology After Their} p-{Value Ban}. \emph{The American Statistician},
\emph{73}(sup1), 374--384.
\url{https://doi.org/10.1080/00031305.2018.1537892}

\leavevmode\vadjust pre{\hypertarget{ref-fried_method_1993}{}}%
Fried, B. J., Boers, M., \& Baker, P. R. (1993). A method for achieving
consensus on rheumatoid arthritis outcome measures: The {OMERACT}
conference process. \emph{The Journal of Rheumatology}, \emph{20}(3),
548--551.

\leavevmode\vadjust pre{\hypertarget{ref-friede_sample_2006}{}}%
Friede, T., \& Kieser, M. (2006). Sample size recalculation in internal
pilot study designs: A review. \emph{Biometrical Journal: Journal of
Mathematical Methods in Biosciences}, \emph{48}(4), 537--555.
\url{https://doi.org/10.1002/bimj.200510238}

\leavevmode\vadjust pre{\hypertarget{ref-fugard_supporting_2015}{}}%
Fugard, A. J. B., \& Potts, H. W. W. (2015). Supporting thinking on
sample sizes for thematic analyses: A quantitative tool.
\emph{International Journal of Social Research Methodology},
\emph{18}(6), 669--684.
\url{https://doi.org/10.1080/13645579.2015.1005453}

\leavevmode\vadjust pre{\hypertarget{ref-funder_evaluating_2019}{}}%
Funder, D. C., \& Ozer, D. J. (2019). Evaluating effect size in
psychological research: {Sense} and nonsense. \emph{Advances in Methods
and Practices in Psychological Science}, \emph{2}(2), 156--168.
\url{https://doi.org/10.1177/2515245919847202}

\leavevmode\vadjust pre{\hypertarget{ref-gerring_mere_2012}{}}%
Gerring, J. (2012). Mere {Description}. \emph{British Journal of
Political Science}, \emph{42}(4), 721--746.
\url{https://doi.org/10.1017/S0007123412000130}

\leavevmode\vadjust pre{\hypertarget{ref-glockner_irrational_2016}{}}%
Glöckner, A. (2016). The irrational hungry judge effect revisited:
{Simulations} reveal that the magnitude of the effect is overestimated.
\emph{Judgment and Decision Making}, \emph{11}(6), 601--610.

\leavevmode\vadjust pre{\hypertarget{ref-glover_likelihood_2004}{}}%
Glover, S., \& Dixon, P. (2004). Likelihood ratios: {A} simple and
flexible statistic for empirical psychologists. \emph{Psychonomic
Bulletin \& Review}, \emph{11}(5), 791--806.

\leavevmode\vadjust pre{\hypertarget{ref-goldacre_compliance_2018}{}}%
Goldacre, B., DeVito, N. J., Heneghan, C., Irving, F., Bacon, S.,
Fleminger, J., \& Curtis, H. (2018). Compliance with requirement to
report results on the {EU Clinical Trials Register}: Cohort study and
web resource. \emph{BMJ}, \emph{362}, k3218.
\url{https://doi.org/10.1136/bmj.k3218}

\leavevmode\vadjust pre{\hypertarget{ref-good_bayesnon-bayes_1992}{}}%
Good, I. J. (1992). The {Bayes}/{Non-Bayes} compromise: {A} brief
review. \emph{Journal of the American Statistical Association},
\emph{87}(419), 597--606. \url{https://doi.org/10.2307/2290192}

\leavevmode\vadjust pre{\hypertarget{ref-goodyear-smith_analysis_2012}{}}%
Goodyear-Smith, F. A., van Driel, M. L., Arroll, B., \& Del Mar, C.
(2012). Analysis of decisions made in meta-analyses of depression
screening and the risk of confirmation bias: {A} case study. \emph{BMC
Medical Research Methodology}, \emph{12}, 76.
\url{https://doi.org/10.1186/1471-2288-12-76}

\leavevmode\vadjust pre{\hypertarget{ref-gosset_application_1904}{}}%
Gosset, W. S. (1904). \emph{The {Application} of the "{Law} of {Error}"
to the {Work} of the {Brewery}} (1 vol 8; pp. 3--16). {Arthur Guinness
\& Son, Ltd.}

\leavevmode\vadjust pre{\hypertarget{ref-gotz_small_2022}{}}%
Götz, F. M., Gosling, S. D., \& Rentfrow, P. J. (2022). Small {Effects}:
{The Indispensable Foundation} for a {Cumulative Psychological Science}.
\emph{Perspectives on Psychological Science}, \emph{17}(1), 205--215.
\url{https://doi.org/10.1177/1745691620984483}

\leavevmode\vadjust pre{\hypertarget{ref-green_simr_2016}{}}%
Green, P., \& MacLeod, C. J. (2016). {SIMR}: An {R} package for power
analysis of generalized linear mixed models by simulation. \emph{Methods
in Ecology and Evolution}, \emph{7}(4), 493--498.
\url{https://doi.org/10.1111/2041-210X.12504}

\leavevmode\vadjust pre{\hypertarget{ref-green_how_1991}{}}%
Green, S. B. (1991). How {Many Subjects Does It Take To Do A Regression
Analysis}. \emph{Multivariate Behavioral Research}, \emph{26}(3),
499--510. \url{https://doi.org/10.1207/s15327906mbr2603_7}

\leavevmode\vadjust pre{\hypertarget{ref-greenland_statistical_2016}{}}%
Greenland, S., Senn, S. J., Rothman, K. J., Carlin, J. B., Poole, C.,
Goodman, S. N., \& Altman, D. G. (2016). Statistical tests, {P} values,
confidence intervals, and power: A guide to misinterpretations.
\emph{European Journal of Epidemiology}, \emph{31}(4), 337--350.
\url{https://doi.org/10.1007/s10654-016-0149-3}

\leavevmode\vadjust pre{\hypertarget{ref-greenwald_consequences_1975}{}}%
Greenwald, A. G. (1975). Consequences of prejudice against the null
hypothesis. \emph{Psychological Bulletin}, \emph{82}(1), 1.

\leavevmode\vadjust pre{\hypertarget{ref-grunwald_safe_2019}{}}%
Grünwald, P., de Heide, R., \& Koolen, W. (2019). Safe {Testing}.
\emph{arXiv:1906.07801 {[}Cs, Math, Stat{]}}.
\url{https://arxiv.org/abs/1906.07801}

\leavevmode\vadjust pre{\hypertarget{ref-gupta_intention_2011}{}}%
Gupta, S. K. (2011). Intention-to-treat concept: {A} review.
\emph{Perspectives in Clinical Research}, \emph{2}(3), 109--112.
\url{https://doi.org/10.4103/2229-3485.83221}

\leavevmode\vadjust pre{\hypertarget{ref-hacking_logic_1965}{}}%
Hacking, I. (1965). \emph{Logic of {Statistical Inference}}. {Cambridge
University Press}.

\leavevmode\vadjust pre{\hypertarget{ref-hagger_multilab_2016}{}}%
Hagger, M. S., Chatzisarantis, N. L. D., Alberts, H., Anggono, C. O.,
Batailler, C., Birt, A. R., Brand, R., Brandt, M. J., Brewer, G.,
Bruyneel, S., Calvillo, D. P., Campbell, W. K., Cannon, P. R., Carlucci,
M., Carruth, N. P., Cheung, T., Crowell, A., De Ridder, D. T. D.,
Dewitte, S., \ldots{} Zwienenberg, M. (2016). A {Multilab Preregistered
Replication} of the {Ego-Depletion Effect}. \emph{Perspectives on
Psychological Science}, \emph{11}(4), 546--573.
\url{https://doi.org/10.1177/1745691616652873}

\leavevmode\vadjust pre{\hypertarget{ref-hallahan_statistical_1996}{}}%
Hallahan, M., \& Rosenthal, R. (1996). Statistical power: {Concepts},
procedures, and applications. \emph{Behaviour Research and Therapy},
\emph{34}(5), 489--499.
\url{https://doi.org/10.1016/0005-7967(95)00082-8}

\leavevmode\vadjust pre{\hypertarget{ref-halpern_sample_2001}{}}%
Halpern, J., Brown Jr, B. W., \& Hornberger, J. (2001). The sample size
for a clinical trial: {A Bayesian} decision theoretic approach.
\emph{Statistics in Medicine}, \emph{20}(6), 841--858.
\url{https://doi.org/10.1002/sim.703}

\leavevmode\vadjust pre{\hypertarget{ref-halpern_continuing_2002}{}}%
Halpern, S. D., Karlawish, J. H., \& Berlin, J. A. (2002). The
continuing unethical conduct of underpowered clinical trials.
\emph{Jama}, \emph{288}(3), 358--362.
\url{https://doi.org/doi:10.1001/jama.288.3.358}

\leavevmode\vadjust pre{\hypertarget{ref-hand_deconstructing_1994}{}}%
Hand, D. J. (1994). Deconstructing {Statistical Questions}.
\emph{Journal of the Royal Statistical Society. Series A (Statistics in
Society)}, \emph{157}(3), 317--356.
\url{https://doi.org/10.2307/2983526}

\leavevmode\vadjust pre{\hypertarget{ref-harms_making_2018}{}}%
Harms, C., \& Lakens, D. (2018). Making 'null effects' informative:
Statistical techniques and inferential frameworks. \emph{Journal of
Clinical and Translational Research}, \emph{3}, 382--393.
\url{https://doi.org/10.18053/jctres.03.2017S2.007}

\leavevmode\vadjust pre{\hypertarget{ref-harrer_doing_2021}{}}%
Harrer, M., Cuijpers, P., Furukawa, T. A., \& Ebert, D. D. (2021).
\emph{Doing {Meta-Analysis} with {R}: {A Hands-On Guide}}. {Chapman and
Hall/CRC}. \url{https://doi.org/10.1201/9781003107347}

\leavevmode\vadjust pre{\hypertarget{ref-hauck_new_1984}{}}%
Hauck, D. W. W., \& Anderson, S. (1984). A new statistical procedure for
testing equivalence in two-group comparative bioavailability trials.
\emph{Journal of Pharmacokinetics and Biopharmaceutics}, \emph{12}(1),
83--91. \url{https://doi.org/10.1007/BF01063612}

\leavevmode\vadjust pre{\hypertarget{ref-hedges_power_2001}{}}%
Hedges, L. V., \& Pigott, T. D. (2001). The power of statistical tests
in meta-analysis. \emph{Psychological Methods}, \emph{6}(3), 203--217.
\url{https://doi.org/10.1037/1082-989X.6.3.203}

\leavevmode\vadjust pre{\hypertarget{ref-hilgard_maximal_2021}{}}%
Hilgard, J. (2021). Maximal positive controls: {A} method for estimating
the largest plausible effect size. \emph{Journal of Experimental Social
Psychology}, \emph{93}. \url{https://doi.org/10.1016/j.jesp.2020.104082}

\leavevmode\vadjust pre{\hypertarget{ref-hill_empirical_2008}{}}%
Hill, C. J., Bloom, H. S., Black, A. R., \& Lipsey, M. W. (2008).
Empirical {Benchmarks} for {Interpreting Effect Sizes} in {Research}.
\emph{Child Development Perspectives}, \emph{2}(3), 172--177.
\url{https://doi.org/10.1111/j.1750-8606.2008.00061.x}

\leavevmode\vadjust pre{\hypertarget{ref-hodges_testing_1954}{}}%
Hodges, J. L., \& Lehmann, E. L. (1954). Testing the {Approximate
Validity} of {Statistical Hypotheses}. \emph{Journal of the Royal
Statistical Society. Series B (Methodological)}, \emph{16}(2), 261--268.
\url{https://doi.org/10.1111/j.2517-6161.1954.tb00169.x}

\leavevmode\vadjust pre{\hypertarget{ref-hoenig_abuse_2001}{}}%
Hoenig, J. M., \& Heisey, D. M. (2001). The abuse of power: The
pervasive fallacy of power calculations for data analysis. \emph{The
American Statistician}, \emph{55}(1), 19--24.
\url{https://doi.org/10.1198/000313001300339897}

\leavevmode\vadjust pre{\hypertarget{ref-huedo-medina_assessing_2006}{}}%
Huedo-Medina, T. B., Sánchez-Meca, J., Marín-Martínez, F., \& Botella,
J. (2006). Assessing heterogeneity in meta-analysis: {Q} statistic or
{I}\$2̂\$ index? \emph{Psychological Methods}, \emph{11}(2), 193.

\leavevmode\vadjust pre{\hypertarget{ref-hung_behavior_1997}{}}%
Hung, H. M. J., O'Neill, R. T., Bauer, P., \& Kohne, K. (1997). The
{Behavior} of the {P-Value When} the {Alternative Hypothesis} is {True}.
\emph{Biometrics}, \emph{53}(1), 11--22.
\url{https://doi.org/10.2307/2533093}

\leavevmode\vadjust pre{\hypertarget{ref-hyde_gender_2008}{}}%
Hyde, J. S., Lindberg, S. M., Linn, M. C., Ellis, A. B., \& Williams, C.
C. (2008). Gender {Similarities Characterize Math Performance}.
\emph{Science}, \emph{321}(5888), 494--495.
\url{https://doi.org/10.1126/science.1160364}

\leavevmode\vadjust pre{\hypertarget{ref-ioannidis_why_2005}{}}%
Ioannidis, J. P. A. (2005). Why {Most Published Research Findings Are
False}. \emph{PLoS Medicine}, \emph{2}(8), e124.
\url{https://doi.org/10.1371/journal.pmed.0020124}

\leavevmode\vadjust pre{\hypertarget{ref-ioannidis_exploratory_2007}{}}%
Ioannidis, J. P. A., \& Trikalinos, T. A. (2007). An exploratory test
for an excess of significant findings. \emph{Clinical Trials},
\emph{4}(3), 245--253. \url{https://doi.org/10.1177/1740774507079441}

\leavevmode\vadjust pre{\hypertarget{ref-iyengar_selection_1988}{}}%
Iyengar, S., \& Greenhouse, J. B. (1988). Selection {Models} and the
{File Drawer Problem}. \emph{Statistical Science}, \emph{3}(1),
109--117.

\leavevmode\vadjust pre{\hypertarget{ref-jaeschke_measurement_1989}{}}%
Jaeschke, R., Singer, J., \& Guyatt, G. H. (1989). Measurement of health
status: {Ascertaining} the minimal clinically important difference.
\emph{Controlled Clinical Trials}, \emph{10}(4), 407--415.
\url{https://doi.org/10.1016/0197-2456(89)90005-6}

\leavevmode\vadjust pre{\hypertarget{ref-jeffreys_theory_1939}{}}%
Jeffreys, H. (1939). \emph{Theory of probability} (1st ed). {Oxford
University Press}.

\leavevmode\vadjust pre{\hypertarget{ref-jennison_group_2000}{}}%
Jennison, C., \& Turnbull, B. W. (2000). \emph{Group sequential methods
with applications to clinical trials}. {Chapman \& Hall/CRC}.

\leavevmode\vadjust pre{\hypertarget{ref-john_measuring_2012}{}}%
John, L. K., Loewenstein, G., \& Prelec, D. (2012). Measuring the
prevalence of questionable research practices with incentives for truth
telling. \emph{Psychological Science}, \emph{23}(5), 524--532.

\leavevmode\vadjust pre{\hypertarget{ref-johnson_revised_2013}{}}%
Johnson, V. E. (2013). Revised standards for statistical evidence.
\emph{Proceedings of the National Academy of Sciences}, \emph{110}(48),
19313--19317. \url{https://doi.org/10.1073/pnas.1313476110}

\leavevmode\vadjust pre{\hypertarget{ref-jones_test_1952}{}}%
Jones, L. V. (1952). Test of hypotheses: One-sided vs. Two-sided
alternatives. \emph{Psychological Bulletin}, \emph{49}(1), 43--46.
https://doi.org/\url{http://dx.doi.org/10.1037/h0056832}

\leavevmode\vadjust pre{\hypertarget{ref-jostmann_weight_2009}{}}%
Jostmann, N. B., Lakens, D., \& Schubert, T. W. (2009). Weight as an
{Embodiment} of {Importance}. \emph{Psychological Science},
\emph{20}(9), 1169--1174.
\url{https://doi.org/10.1111/j.1467-9280.2009.02426.x}

\leavevmode\vadjust pre{\hypertarget{ref-jostmann_short_2016}{}}%
Jostmann, N. B., Lakens, D., \& Schubert, T. W. (2016). A short history
of the weight-importance effect and a recommendation for pre-testing:
{Commentary} on {Ebersole} et al. (2016). \emph{Journal of Experimental
Social Psychology}, \emph{67}, 93--94.
\url{https://doi.org/10.1016/j.jesp.2015.12.001}

\leavevmode\vadjust pre{\hypertarget{ref-julious_sample_2004}{}}%
Julious, S. A. (2004). Sample sizes for clinical trials with normal
data. \emph{Statistics in Medicine}, \emph{23}(12), 1921--1986.
\url{https://doi.org/10.1002/sim.1783}

\leavevmode\vadjust pre{\hypertarget{ref-kaplan_likelihood_2015}{}}%
Kaplan, R. M., \& Irvin, V. L. (2015). Likelihood of {Null Effects} of
{Large NHLBI Clinical Trials Has Increased} over {Time}. \emph{PLOS
ONE}, \emph{10}(8), e0132382.
\url{https://doi.org/10.1371/journal.pone.0132382}

\leavevmode\vadjust pre{\hypertarget{ref-kass_bayes_1995}{}}%
Kass, R. E., \& Raftery, A. E. (1995). Bayes factors. \emph{Journal of
the American Statistical Association}, \emph{90}(430), 773--795.
\url{https://doi.org/10.1080/01621459.1995.10476572}

\leavevmode\vadjust pre{\hypertarget{ref-keefe_defining_2013}{}}%
Keefe, R. S. E., Kraemer, H. C., Epstein, R. S., Frank, E., Haynes, G.,
Laughren, T. P., Mcnulty, J., Reed, S. D., Sanchez, J., \& Leon, A. C.
(2013).
\href{https://www.ncbi.nlm.nih.gov/pmc/articles/PMC3719483}{Defining a
{Clinically Meaningful Effect} for the {Design} and {Interpretation} of
{Randomized Controlled Trials}}. \emph{Innovations in Clinical
Neuroscience}, \emph{10}(5-6 Suppl A), 4S--19S.

\leavevmode\vadjust pre{\hypertarget{ref-kelley_confidence_2007}{}}%
Kelley, K. (2007). Confidence {Intervals} for {Standardized Effect
Sizes}: {Theory}, {Application}, and {Implementation}. \emph{Journal of
Statistical Software}, \emph{20}(8).
\url{https://doi.org/10.18637/JSS.V020.I08}

\leavevmode\vadjust pre{\hypertarget{ref-kelley_effect_2012}{}}%
Kelley, K., \& Preacher, K. J. (2012). On effect size.
\emph{Psychological Methods}, \emph{17}(2), 137--152.
\url{https://doi.org/10.1037/a0028086}

\leavevmode\vadjust pre{\hypertarget{ref-kelley_sample_2006}{}}%
Kelley, K., \& Rausch, J. R. (2006). Sample size planning for the
standardized mean difference: Accuracy in parameter estimation via
narrow confidence intervals. \emph{Psychological Methods}, \emph{11}(4),
363--385. \url{https://doi.org/10.1037}

\leavevmode\vadjust pre{\hypertarget{ref-kenett_information_2016}{}}%
Kenett, R. S., Shmueli, G., \& Kenett, R. (2016). \emph{Information
{Quality}: {The Potential} of {Data} and {Analytics} to {Generate
Knowledge}} (1st edition). {Wiley}.

\leavevmode\vadjust pre{\hypertarget{ref-kennedy-shaffer_before_2019}{}}%
Kennedy-Shaffer, L. (2019). Before p {\(<\)} 0.05 to {Beyond} p {\(<\)}
0.05: {Using History} to {Contextualize} p-{Values} and {Significance
Testing}. \emph{The American Statistician}, \emph{73}(sup1), 82--90.
\url{https://doi.org/10.1080/00031305.2018.1537891}

\leavevmode\vadjust pre{\hypertarget{ref-kenny_unappreciated_2019}{}}%
Kenny, D. A., \& Judd, C. M. (2019). The unappreciated heterogeneity of
effect sizes: {Implications} for power, precision, planning of research,
and replication. \emph{Psychological Methods}, \emph{24}(5), 578--589.
\url{https://doi.org/10.1037/met0000209}

\leavevmode\vadjust pre{\hypertarget{ref-keppel_design_1991}{}}%
Keppel, G. (1991). \emph{Design and analysis: {A} researcher's handbook,
3rd ed} (pp. xiii, 594). {Prentice-Hall, Inc}.

\leavevmode\vadjust pre{\hypertarget{ref-kerr_harking_1998}{}}%
Kerr, N. L. (1998). {HARKing}: {Hypothesizing After} the {Results} are
{Known}. \emph{Personality and Social Psychology Review}, \emph{2}(3),
196--217. \url{https://doi.org/10.1207/s15327957pspr0203_4}

\leavevmode\vadjust pre{\hypertarget{ref-king_point_2011}{}}%
King, M. T. (2011). A point of minimal important difference ({MID}): A
critique of terminology and methods. \emph{Expert Review of
Pharmacoeconomics \& Outcomes Research}, \emph{11}(2), 171--184.
\url{https://doi.org/10.1586/erp.11.9}

\leavevmode\vadjust pre{\hypertarget{ref-kish_survey_1965}{}}%
Kish, L. (1965). \emph{Survey {Sampling}}. {Wiley}.

\leavevmode\vadjust pre{\hypertarget{ref-kraft_interpreting_2020}{}}%
Kraft, M. A. (2020). Interpreting effect sizes of education
interventions. \emph{Educational Researcher}, \emph{49}(4), 241--253.
\url{https://doi.org/10.3102/0013189X20912798}

\leavevmode\vadjust pre{\hypertarget{ref-kruschke_rejecting_2018}{}}%
Kruschke, J. K. (2018). Rejecting or {Accepting Parameter Values} in
{Bayesian Estimation}. \emph{Advances in Methods and Practices in
Psychological Science}, \emph{1}(2), 270--280.
\url{https://doi.org/10.1177/2515245918771304}

\leavevmode\vadjust pre{\hypertarget{ref-kruschke_doing_2014}{}}%
Kruschke, J. K. (2014). \emph{Doing {Bayesian Data Analysis}, {Second
Edition}: {A Tutorial} with {R}, {JAGS}, and {Stan}} (2 edition).
{Academic Press}.

\leavevmode\vadjust pre{\hypertarget{ref-kruschke_bayesian_2011}{}}%
Kruschke, J. K. (2011). Bayesian assessment of null values via parameter
estimation and model comparison. \emph{Perspectives on Psychological
Science}, \emph{6}(3), 299--312.

\leavevmode\vadjust pre{\hypertarget{ref-kruschke_bayesian_2013}{}}%
Kruschke, J. K. (2013). Bayesian estimation supersedes the t test.
\emph{Journal of Experimental Psychology: General}, \emph{142}(2),
573--603. \url{https://doi.org/10.1037/a0029146}

\leavevmode\vadjust pre{\hypertarget{ref-kruschke_bayesian_2017}{}}%
Kruschke, J. K., \& Liddell, T. M. (2017). The {Bayesian New
Statistics}: {Hypothesis} testing, estimation, meta-analysis, and power
analysis from a {Bayesian} perspective. \emph{Psychonomic Bulletin \&
Review}. \url{https://doi.org/10.3758/s13423-016-1221-4}

\leavevmode\vadjust pre{\hypertarget{ref-R-BEST}{}}%
Kruschke, J. K., \& Meredith, M. (2021). \emph{BEST: Bayesian estimation
supersedes the t-test}. \url{https://CRAN.R-project.org/package=BEST}

\leavevmode\vadjust pre{\hypertarget{ref-lakens_why_2022}{}}%
Lakens, D. (2022a). Why {P} values are not measures of evidence.
\emph{Trends in Ecology \& Evolution}.
\url{https://doi.org/10.1016/j.tree.2021.12.006}

\leavevmode\vadjust pre{\hypertarget{ref-lakens_practical_2021}{}}%
Lakens, D. (2021). The practical alternative to the p value is the
correctly used p value. \emph{Perspectives on Psychological Science},
\emph{16}(3), 639--648. \url{https://doi.org/10.1177/1745691620958012}

\leavevmode\vadjust pre{\hypertarget{ref-lakens_calculating_2013}{}}%
Lakens, D. (2013). Calculating and reporting effect sizes to facilitate
cumulative science: A practical primer for t-tests and {ANOVAs}.
\emph{Frontiers in Psychology}, \emph{4}.
\url{https://doi.org/10.3389/fpsyg.2013.00863}

\leavevmode\vadjust pre{\hypertarget{ref-lakens_performing_2014}{}}%
Lakens, D. (2014). Performing high-powered studies efficiently with
sequential analyses: {Sequential} analyses. \emph{European Journal of
Social Psychology}, \emph{44}(7), 701--710.
\url{https://doi.org/10.1002/ejsp.2023}

\leavevmode\vadjust pre{\hypertarget{ref-lakens_challenges_2015}{}}%
Lakens, D. (2015). On the challenges of drawing conclusions from
{\emph{p}} -values just below 0.05. \emph{PeerJ}, \emph{3}, e1142.
\url{https://doi.org/10.7717/peerj.1142}

\leavevmode\vadjust pre{\hypertarget{ref-lakens_equivalence_2017}{}}%
Lakens, D. (2017). Equivalence {Tests}: {A Practical Primer} for t
{Tests}, {Correlations}, and {Meta-Analyses}. \emph{Social Psychological
and Personality Science}, \emph{8}(4), 355--362.
\url{https://doi.org/10.1177/1948550617697177}

\leavevmode\vadjust pre{\hypertarget{ref-lakens_value_2019}{}}%
Lakens, D. (2019). The value of preregistration for psychological
science: {A} conceptual analysis. \emph{Japanese Psychological Review},
\emph{62}(3), 221--230. \url{https://doi.org/10.24602/sjpr.62.3_221}

\leavevmode\vadjust pre{\hypertarget{ref-lakens_sample_2022}{}}%
Lakens, D. (2022b). Sample {Size Justification}. \emph{Collabra:
Psychology}. \url{https://doi.org/10.31234/osf.io/9d3yf}

\leavevmode\vadjust pre{\hypertarget{ref-lakens_justify_2018}{}}%
Lakens, D., Adolfi, F. G., Albers, C. J., Anvari, F., Apps, M. A. J.,
Argamon, S. E., Baguley, T., Becker, R. B., Benning, S. D., Bradford, D.
E., Buchanan, E. M., Caldwell, A. R., Calster, B., Carlsson, R., Chen,
S.-C., Chung, B., Colling, L. J., Collins, G. S., Crook, Z., \ldots{}
Zwaan, R. A. (2018). Justify your alpha. \emph{Nature Human Behaviour},
\emph{2}, 168--171. \url{https://doi.org/10.1038/s41562-018-0311-x}

\leavevmode\vadjust pre{\hypertarget{ref-R-TOSTER}{}}%
Lakens, D., \& Caldwell, A. (2022). \emph{TOSTER: Two one-sided tests
(TOST) equivalence testing}. \url{https://aaroncaldwell.us/TOSTERpkg/}

\leavevmode\vadjust pre{\hypertarget{ref-lakens_simulation-based_2021}{}}%
Lakens, D., \& Caldwell, A. R. (2021). Simulation-{Based Power Analysis}
for {Factorial Analysis} of {Variance Designs}. \emph{Advances in
Methods and Practices in Psychological Science}, \emph{4}(1),
2515245920951503. \url{https://doi.org/10.1177/2515245920951503}

\leavevmode\vadjust pre{\hypertarget{ref-lakens_improving_2020-2}{}}%
Lakens, D., \& DeBruine, L. (2020). \emph{Improving {Transparency},
{Falsifiability}, and {Rigour} by {Making Hypothesis Tests Machine
Readable}}. \url{https://doi.org/10.31234/osf.io/5xcda}

\leavevmode\vadjust pre{\hypertarget{ref-lakens_too_2017}{}}%
Lakens, D., \& Etz, A. J. (2017). Too {True} to be {Bad}: {When Sets} of
{Studies With Significant} and {Nonsignificant Findings Are Probably
True}. \emph{Social Psychological and Personality Science}, \emph{8}(8),
875--881. \url{https://doi.org/10.1177/1948550617693058}

\leavevmode\vadjust pre{\hypertarget{ref-lakens_reproducibility_2016}{}}%
Lakens, D., Hilgard, J., \& Staaks, J. (2016). On the reproducibility of
meta-analyses: Six practical recommendations. \emph{BMC Psychology},
\emph{4}, 24. \url{https://doi.org/10.1186/s40359-016-0126-3}

\leavevmode\vadjust pre{\hypertarget{ref-lakens_improving_2020}{}}%
Lakens, D., McLatchie, N., Isager, P. M., Scheel, A. M., \& Dienes, Z.
(2020). Improving {Inferences About Null Effects With Bayes Factors} and
{Equivalence Tests}. \emph{The Journals of Gerontology: Series B},
\emph{75}(1), 45--57. \url{https://doi.org/10.1093/geronb/gby065}

\leavevmode\vadjust pre{\hypertarget{ref-lakens_equivalence_2018}{}}%
Lakens, D., Scheel, A. M., \& Isager, P. M. (2018). Equivalence testing
for psychological research: {A} tutorial. \emph{Advances in Methods and
Practices in Psychological Science}, \emph{1}(2), 259--269.
\url{https://doi.org/10.1177/2515245918770963}

\leavevmode\vadjust pre{\hypertarget{ref-lan_discrete_1983}{}}%
Lan, K. K. G., \& DeMets, D. L. (1983). Discrete {Sequential Boundaries}
for {Clinical Trials}. \emph{Biometrika}, \emph{70}(3), 659.
\url{https://doi.org/10.2307/2336502}

\leavevmode\vadjust pre{\hypertarget{ref-lawrence_lesson_2021}{}}%
Lawrence, J. M., Meyerowitz-Katz, G., Heathers, J. A. J., Brown, N. J.
L., \& Sheldrick, K. A. (2021). The lesson of ivermectin: Meta-analyses
based on summary data alone are inherently unreliable. \emph{Nature
Medicine}, \emph{27}(11), 1853--1854.
\url{https://doi.org/10.1038/s41591-021-01535-y}

\leavevmode\vadjust pre{\hypertarget{ref-leamer_specification_1978}{}}%
Leamer, E. E. (1978). \emph{Specification {Searches}: {Ad Hoc Inference}
with {Nonexperimental Data}} (1 edition). {Wiley}.

\leavevmode\vadjust pre{\hypertarget{ref-lehmann_testing_2005}{}}%
Lehmann, E. L., \& Romano, J. P. (2005). \emph{Testing statistical
hypotheses} (3rd ed). {Springer}.

\leavevmode\vadjust pre{\hypertarget{ref-lenth_practical_2001}{}}%
Lenth, R. V. (2001). Some practical guidelines for effective sample size
determination. \emph{The American Statistician}, \emph{55}(3), 187--193.
\url{https://doi.org/10.1198/000313001317098149}

\leavevmode\vadjust pre{\hypertarget{ref-lenth_post_2007}{}}%
Lenth, R. V. (2007). Post hoc power: Tables and commentary. \emph{Iowa
City: Department of Statistics and Actuarial Science, University of
Iowa}.

\leavevmode\vadjust pre{\hypertarget{ref-leon_role_2011}{}}%
Leon, A. C., Davis, L. L., \& Kraemer, H. C. (2011). The {Role} and
{Interpretation} of {Pilot Studies} in {Clinical Research}.
\emph{Journal of Psychiatric Research}, \emph{45}(5), 626--629.
\url{https://doi.org/10.1016/j.jpsychires.2010.10.008}

\leavevmode\vadjust pre{\hypertarget{ref-levine_communication_2008}{}}%
Levine, T. R., Weber, R., Park, H. S., \& Hullett, C. R. (2008). A
communication researchers' guide to null hypothesis significance testing
and alternatives. \emph{Human Communication Research}, \emph{34}(2),
188--209.

\leavevmode\vadjust pre{\hypertarget{ref-leys_how_2019}{}}%
Leys, C., Delacre, M., Mora, Y. L., Lakens, D., \& Ley, C. (2019). How
to {Classify}, {Detect}, and {Manage Univariate} and {Multivariate
Outliers}, {With Emphasis} on {Pre-Registration}. \emph{International
Review of Social Psychology}, \emph{32}(1), 5.
\url{https://doi.org/10.5334/irsp.289}

\leavevmode\vadjust pre{\hypertarget{ref-linden_heterogeneity_2021}{}}%
Linden, A. H., \& Hönekopp, J. (2021). Heterogeneity of {Research
Results}: {A New Perspective From Which} to {Assess} and {Promote
Progress} in {Psychological Science}. \emph{Perspectives on
Psychological Science}, \emph{16}(2), 358--376.
\url{https://doi.org/10.1177/1745691620964193}

\leavevmode\vadjust pre{\hypertarget{ref-lindley_statistical_1957}{}}%
Lindley, D. V. (1957). A statistical paradox. \emph{Biometrika},
\emph{44}(1/2), 187--192.

\leavevmode\vadjust pre{\hypertarget{ref-lindsay_replication_2015}{}}%
Lindsay, D. S. (2015). Replication in {Psychological Science}.
\emph{Psychological Science}, \emph{26}(12), 1827--1832.
\url{https://doi.org/10.1177/0956797615616374}

\leavevmode\vadjust pre{\hypertarget{ref-lovakov_empirically_2017}{}}%
Lovakov, A., \& Agadullina, E. (2017). Empirically {Derived Guidelines}
for {Interpreting Effect Size} in {Social Psychology}. \emph{PsyArXiv}.
\url{https://doi.org/10.17605/OSF.IO/2EPC4}

\leavevmode\vadjust pre{\hypertarget{ref-R-JustifyAlpha}{}}%
Maier, M., \& Lakens, D. (2021). \emph{JustifyAlpha: Justifying alpha
levels for hypothesis tests}.
\url{https://CRAN.R-project.org/package=JustifyAlpha}

\leavevmode\vadjust pre{\hypertarget{ref-maier_justify_2022}{}}%
Maier, M., \& Lakens, D. (2022). Justify your alpha: {A} primer on two
practical approaches. \emph{Advances in Methods and Practices in
Psychological Science}. \url{https://doi.org/10.31234/osf.io/ts4r6}

\leavevmode\vadjust pre{\hypertarget{ref-makel_both_2021}{}}%
Makel, M. C., Hodges, J., Cook, B. G., \& Plucker, J. A. (2021). Both
{Questionable} and {Open Research Practices Are Prevalent} in {Education
Research}. \emph{Educational Researcher}, \emph{50}(8), 493--504.
\url{https://doi.org/10.3102/0013189X211001356}

\leavevmode\vadjust pre{\hypertarget{ref-marshall_does_2013}{}}%
Marshall, B., Cardon, P., Poddar, A., \& Fontenot, R. (2013). Does
{Sample Size Matter} in {Qualitative Research}?: {A Review} of
{Qualitative Interviews} in is {Research}. \emph{Journal of Computer
Information Systems}, \emph{54}(1), 11--22.
\url{https://doi.org/10.1080/08874417.2013.11645667}

\leavevmode\vadjust pre{\hypertarget{ref-maxwell_designing_2004}{}}%
Maxwell, S. E., \& Delaney, H. D. (2004). \emph{Designing experiments
and analyzing data: A model comparison perspective} (2nd ed). {Lawrence
Erlbaum Associates}.

\leavevmode\vadjust pre{\hypertarget{ref-maxwell_designing_2017}{}}%
Maxwell, S. E., Delaney, H. D., \& Kelley, K. (2017). \emph{Designing
{Experiments} and {Analyzing Data}: {A Model Comparison Perspective},
{Third Edition}} (3 edition). {Routledge}.

\leavevmode\vadjust pre{\hypertarget{ref-maxwell_ethics_2011}{}}%
Maxwell, S. E., \& Kelley, K. (2011). Ethics and sample size planning.
In \emph{Handbook of ethics in quantitative methodology} (pp. 179--204).
{Routledge}.

\leavevmode\vadjust pre{\hypertarget{ref-maxwell_sample_2008}{}}%
Maxwell, S. E., Kelley, K., \& Rausch, J. R. (2008). Sample {Size
Planning} for {Statistical Power} and {Accuracy} in {Parameter
Estimation}. \emph{Annual Review of Psychology}, \emph{59}(1), 537--563.
\url{https://doi.org/10.1146/annurev.psych.59.103006.093735}

\leavevmode\vadjust pre{\hypertarget{ref-mayo_statistical_2018}{}}%
Mayo, D. G. (2018). \emph{Statistical inference as severe testing: How
to get beyond the statistics wars}. {Cambridge University Press}.

\leavevmode\vadjust pre{\hypertarget{ref-mazzolari_myths_2022}{}}%
Mazzolari, R., Porcelli, S., Bishop, D. J., \& Lakens, D. (2022). Myths
and methodologies: {The} use of equivalence and non-inferiority tests
for interventional studies in exercise physiology and sport science.
\emph{Experimental Physiology}, \emph{107}(3), 201--212.
\url{https://doi.org/10.1113/EP090171}

\leavevmode\vadjust pre{\hypertarget{ref-mccarthy_registered_2018}{}}%
McCarthy, R. J., Skowronski, J. J., Verschuere, B., Meijer, E. H., Jim,
A., Hoogesteyn, K., Orthey, R., Acar, O. A., Aczel, B., Bakos, B. E.,
Barbosa, F., Baskin, E., Bègue, L., Ben-Shakhar, G., Birt, A. R., Blatz,
L., Charman, S. D., Claesen, A., Clay, S. L., \ldots{} Yıldız, E.
(2018). Registered {Replication Report} on {Srull} and {Wyer} (1979).
\emph{Advances in Methods and Practices in Psychological Science},
\emph{1}(3), 321--336. \url{https://doi.org/10.1177/2515245918777487}

\leavevmode\vadjust pre{\hypertarget{ref-mcelreath_statistical_2016}{}}%
McElreath, R. (2016). \emph{Statistical {Rethinking}: {A Bayesian
Course} with {Examples} in {R} and {Stan}} (Vol. 122). {CRC Press}.

\leavevmode\vadjust pre{\hypertarget{ref-mcgrath_when_2006}{}}%
McGrath, R. E., \& Meyer, G. J. (2006). When effect sizes disagree:
{The} case of r and d. \emph{Psychological Methods}, \emph{11}(4),
386--401. \url{https://doi.org/10.1037/1082-989X.11.4.386}

\leavevmode\vadjust pre{\hypertarget{ref-mcgraw_common_1992}{}}%
McGraw, K. O., \& Wong, S. P. (1992). A common language effect size
statistic. \emph{Psychological Bulletin}, \emph{111}(2), 361--365.
\url{https://doi.org/10.1037/0033-2909.111.2.361}

\leavevmode\vadjust pre{\hypertarget{ref-mcintosh_power_2020}{}}%
McIntosh, R. D., \& Rittmo, J. Ö. (2020). \emph{Power calculations in
single case neuropsychology}. {PsyArXiv}.
\url{https://doi.org/10.31234/osf.io/fxz49}

\leavevmode\vadjust pre{\hypertarget{ref-meehl_theoretical_1978}{}}%
Meehl, P. E. (1978). Theoretical {Risks} and {Tabular Asterisks}: {Sir
Karl}, {Sir Ronald}, and the {Slow Progress} of {Soft Psychology}.
\emph{Journal of Consulting and Clinical Psychology}, \emph{46}(4),
806--834. \url{https://doi.org/10.1037/0022-006X.46.4.806}

\leavevmode\vadjust pre{\hypertarget{ref-meehl_appraising_1990}{}}%
Meehl, P. E. (1990). Appraising and amending theories: {The} strategy of
{Lakatosian} defense and two principles that warrant it.
\emph{Psychological Inquiry}, \emph{1}(2), 108--141.
\url{https://doi.org/10.1207/s15327965pli0102_1}

\leavevmode\vadjust pre{\hypertarget{ref-R-truncnorm}{}}%
Mersmann, O., Trautmann, H., Steuer, D., \& Bornkamp, B. (2018).
\emph{Truncnorm: Truncated normal distribution}.
\url{https://github.com/olafmersmann/truncnorm}

\leavevmode\vadjust pre{\hypertarget{ref-meyners_equivalence_2012}{}}%
Meyners, M. (2012). Equivalence tests \textendash{} {A} review.
\emph{Food Quality and Preference}, \emph{26}(2), 231--245.
\url{https://doi.org/10.1016/j.foodqual.2012.05.003}

\leavevmode\vadjust pre{\hypertarget{ref-meyvis_increasing_2018}{}}%
Meyvis, T., \& Van Osselaer, S. M. J. (2018). Increasing the {Power} of
{Your Study} by {Increasing} the {Effect Size}. \emph{Journal of
Consumer Research}, \emph{44}(5), 1157--1173.
\url{https://doi.org/10.1093/jcr/ucx110}

\leavevmode\vadjust pre{\hypertarget{ref-millar_maximum_2011}{}}%
Millar, R. B. (2011). \emph{Maximum likelihood estimation and inference:
With examples in {R}, {SAS}, and {ADMB}}. {Wiley}.

\leavevmode\vadjust pre{\hypertarget{ref-miller_what_2009}{}}%
Miller, J. (2009). What is the probability of replicating a
statistically significant effect? \emph{Psychonomic Bulletin \& Review},
\emph{16}(4), 617--640. \url{https://doi.org/10.3758/PBR.16.4.617}

\leavevmode\vadjust pre{\hypertarget{ref-miller_quest_2019}{}}%
Miller, J., \& Ulrich, R. (2019). The quest for an optimal alpha.
\emph{PLOS ONE}, \emph{14}(1), e0208631.
\url{https://doi.org/10.1371/journal.pone.0208631}

\leavevmode\vadjust pre{\hypertarget{ref-morey_power_2020}{}}%
Morey, R. D. (2020). \emph{Power and precision} {[}Blog{]}.
https://medium.com/@richarddmorey/power-and-precision-47f644ddea5e.

\leavevmode\vadjust pre{\hypertarget{ref-morey_pre-registered_2021}{}}%
Morey, R. D., Kaschak, M. P., Díez-Álamo, A. M., Glenberg, A. M., Zwaan,
R. A., Lakens, D., Ibáñez, A., García, A., Gianelli, C., Jones, J. L.,
Madden, J., Alifano, F., Bergen, B., Bloxsom, N. G., Bub, D. N., Cai, Z.
G., Chartier, C. R., Chatterjee, A., Conwell, E., \ldots{} Ziv-Crispel,
N. (2021). A pre-registered, multi-lab non-replication of the
action-sentence compatibility effect ({ACE}). \emph{Psychonomic Bulletin
\& Review}. \url{https://doi.org/10.3758/s13423-021-01927-8}

\leavevmode\vadjust pre{\hypertarget{ref-morris_using_2019}{}}%
Morris, T. P., White, I. R., \& Crowther, M. J. (2019). Using simulation
studies to evaluate statistical methods. \emph{Statistics in Medicine},
\emph{38}(11), 2074--2102. \url{https://doi.org/10.1002/sim.8086}

\leavevmode\vadjust pre{\hypertarget{ref-morse_significance_1995}{}}%
Morse, J. M. (1995). The {Significance} of {Saturation}.
\emph{Qualitative Health Research}, \emph{5}(2), 147--149.
\url{https://doi.org/10.1177/104973239500500201}

\leavevmode\vadjust pre{\hypertarget{ref-moshontz_psychological_2018}{}}%
Moshontz, H., Campbell, L., Ebersole, C. R., IJzerman, H., Urry, H. L.,
Forscher, P. S., Grahe, J. E., McCarthy, R. J., Musser, E. D., \&
Antfolk, J. (2018). The {Psychological Science Accelerator}: {Advancing}
psychology through a distributed collaborative network. \emph{Advances
in Methods and Practices in Psychological Science}, \emph{1}(4),
501--515. \url{https://doi.org/10.1177/2515245918797607}

\leavevmode\vadjust pre{\hypertarget{ref-mrozek_what_2002}{}}%
Mrozek, J. R., \& Taylor, L. O. (2002). What determines the value of
life? A meta-analysis. \emph{Journal of Policy Analysis and Management},
\emph{21}(2), 253--270. \url{https://doi.org/10.1002/pam.10026}

\leavevmode\vadjust pre{\hypertarget{ref-mudge_setting_2012}{}}%
Mudge, J. F., Baker, L. F., Edge, C. B., \& Houlahan, J. E. (2012).
Setting an {Optimal} {\(\alpha\)} {That Minimizes Errors} in {Null
Hypothesis Significance Tests}. \emph{PLOS ONE}, \emph{7}(2), e32734.
\url{https://doi.org/10.1371/journal.pone.0032734}

\leavevmode\vadjust pre{\hypertarget{ref-mullan_town_1985}{}}%
Mullan, F., \& Jacoby, I. (1985). The town meeting for technology: {The}
maturation of consensus conferences. \emph{JAMA}, \emph{254}(8),
1068--1072. \url{https://doi.org/10.1001/jama.1985.03360080080035}

\leavevmode\vadjust pre{\hypertarget{ref-murphy_testing_1999}{}}%
Murphy, K. R., \& Myors, B. (1999). Testing the hypothesis that
treatments have negligible effects: {Minimum-effect} tests in the
general linear model. \emph{Journal of Applied Psychology},
\emph{84}(2), 234--248. \url{https://doi.org/10.1037/0021-9010.84.2.234}

\leavevmode\vadjust pre{\hypertarget{ref-murphy_statistical_2014}{}}%
Murphy, K. R., Myors, B., \& Wolach, A. H. (2014). \emph{Statistical
power analysis: A simple and general model for traditional and modern
hypothesis tests} (Fourth edition). {Routledge, Taylor \& Francis
Group}.

\leavevmode\vadjust pre{\hypertarget{ref-neyman_inductive_1957}{}}%
Neyman, J. (1957). "{Inductive Behavior}" as a {Basic Concept} of
{Philosophy} of {Science}. \emph{Revue de l'Institut International de
Statistique / Review of the International Statistical Institute},
\emph{25}(1/3), 7. \url{https://doi.org/10.2307/1401671}

\leavevmode\vadjust pre{\hypertarget{ref-neyman_problem_1933}{}}%
Neyman, J., \& Pearson, E. S. (1933). On the problem of the most
efficient tests of statistical hypotheses. \emph{Philosophical
Transactions of the Royal Society of London A: Mathematical, Physical
and Engineering Sciences}, \emph{231}(694-706), 289--337.
\url{https://doi.org/10.1098/rsta.1933.0009}

\leavevmode\vadjust pre{\hypertarget{ref-nickerson_null_2000}{}}%
Nickerson, R. S. (2000). Null hypothesis significance testing: {A}
review of an old and continuing controversy. \emph{Psychological
Methods}, \emph{5}(2), 241--301.
\url{https://doi.org/10.1037//1082-989X.5.2.241}

\leavevmode\vadjust pre{\hypertarget{ref-niiniluoto_verisimilitude_1998}{}}%
Niiniluoto, I. (1998). Verisimilitude: {The Third Period}. \emph{The
British Journal for the Philosophy of Science}, \emph{49}, 1--29.

\leavevmode\vadjust pre{\hypertarget{ref-norman_truly_2004}{}}%
Norman, G. R., Sloan, J. A., \& Wyrwich, K. W. (2004). The truly
remarkable universality of half a standard deviation: Confirmation
through another look. \emph{Expert Review of Pharmacoeconomics \&
Outcomes Research}, \emph{4}(5), 581--585.

\leavevmode\vadjust pre{\hypertarget{ref-nosek_registered_2014}{}}%
Nosek, B. A., \& Lakens, D. (2014). Registered reports: {A} method to
increase the credibility of published results. \emph{Social Psychology},
\emph{45}(3), 137--141. \url{https://doi.org/10.1027/1864-9335/a000192}

\leavevmode\vadjust pre{\hypertarget{ref-nuijten_prevalence_2015}{}}%
Nuijten, M. B., Hartgerink, C. H. J., van Assen, M. A. L. M., Epskamp,
S., \& Wicherts, J. M. (2015). The prevalence of statistical reporting
errors in psychology (1985\textendash 2013). \emph{Behavior Research
Methods}. \url{https://doi.org/10.3758/s13428-015-0664-2}

\leavevmode\vadjust pre{\hypertarget{ref-nunnally_place_1960}{}}%
Nunnally, J. (1960). The place of statistics in psychology.
\emph{Educational and Psychological Measurement}, \emph{20}(4),
641--650. \url{https://doi.org/10.1177/001316446002000401}

\leavevmode\vadjust pre{\hypertarget{ref-odonnell_registered_2018}{}}%
O'Donnell, M., Nelson, L. D., Ackermann, E., Aczel, B., Akhtar, A.,
Aldrovandi, S., Alshaif, N., Andringa, R., Aveyard, M., Babincak, P.,
Balatekin, N., Baldwin, S. A., Banik, G., Baskin, E., Bell, R.,
Białobrzeska, O., Birt, A. R., Boot, W. R., Braithwaite, S. R., \ldots{}
Zrubka, M. (2018). Registered {Replication Report}: {Dijksterhuis} and
van {Knippenberg} (1998). \emph{Perspectives on Psychological Science},
\emph{13}(2), 268--294. \url{https://doi.org/10.1177/1745691618755704}

\leavevmode\vadjust pre{\hypertarget{ref-obels_analysis_2020}{}}%
Obels, P., Lakens, D., Coles, N. A., Gottfried, J., \& Green, S. A.
(2020). Analysis of {Open Data} and {Computational Reproducibility} in
{Registered Reports} in {Psychology}. \emph{Advances in Methods and
Practices in Psychological Science}, \emph{3}(2), 229--237.
\url{https://doi.org/10.1177/2515245920918872}

\leavevmode\vadjust pre{\hypertarget{ref-okada_is_2013}{}}%
Okada, K. (2013). Is {Omega Squared Less Biased}? A {Comparison} of
{Three Major Effect Size Indices} in {One-Way Anova}.
\emph{Behaviormetrika}, \emph{40}(2), 129--147.
\url{https://doi.org/10.2333/bhmk.40.129}

\leavevmode\vadjust pre{\hypertarget{ref-olejnik_generalized_2003}{}}%
Olejnik, S., \& Algina, J. (2003). Generalized {Eta} and {Omega Squared
Statistics}: {Measures} of {Effect Size} for {Some Common Research
Designs}. \emph{Psychological Methods}, \emph{8}(4), 434--447.
\url{https://doi.org/10.1037/1082-989X.8.4.434}

\leavevmode\vadjust pre{\hypertarget{ref-olsson-collentine_heterogeneity_2020}{}}%
Olsson-Collentine, A., Wicherts, J. M., \& van Assen, M. A. L. M.
(2020). Heterogeneity in direct replications in psychology and its
association with effect size. \emph{Psychological Bulletin},
\emph{146}(10), 922--940. \url{https://doi.org/10.1037/bul0000294}

\leavevmode\vadjust pre{\hypertarget{ref-Gifski2021}{}}%
Ooms, J. (2021). \emph{Gifski: Highest quality GIF encoder}.
\url{https://CRAN.R-project.org/package=gifski}

\leavevmode\vadjust pre{\hypertarget{ref-open_science_collaboration_estimating_2015}{}}%
Open Science Collaboration. (2015). Estimating the reproducibility of
psychological science. \emph{Science}, \emph{349}(6251),
aac4716--aac4716. \url{https://doi.org/10.1126/science.aac4716}

\leavevmode\vadjust pre{\hypertarget{ref-orben_crud_2020}{}}%
Orben, A., \& Lakens, D. (2020). Crud ({Re}){Defined}. \emph{Advances in
Methods and Practices in Psychological Science}, \emph{3}(2), 238--247.
\url{https://doi.org/10.1177/2515245920917961}

\leavevmode\vadjust pre{\hypertarget{ref-parker_sample_2003}{}}%
Parker, R. A., \& Berman, N. G. (2003). Sample {Size}. \emph{The
American Statistician}, \emph{57}(3), 166--170.
\url{https://doi.org/10.1198/0003130031919}

\leavevmode\vadjust pre{\hypertarget{ref-parkhurst_statistical_2001}{}}%
Parkhurst, D. F. (2001). Statistical significance tests: {Equivalence}
and reverse tests should reduce misinterpretation. \emph{Bioscience},
\emph{51}(12), 1051--1057.
\url{https://doi.org/10.1641/0006-3568(2001)051\%5B1051:SSTEAR\%5D2.0.CO;2}

\leavevmode\vadjust pre{\hypertarget{ref-parsons_psychological_2019}{}}%
Parsons, S., Kruijt, A.-W., \& Fox, E. (2019). Psychological {Science
Needs} a {Standard Practice} of {Reporting} the {Reliability} of
{Cognitive-Behavioral Measurements}. \emph{Advances in Methods and
Practices in Psychological Science}, \emph{2}(4), 378--395.
\url{https://doi.org/10.1177/2515245919879695}

\leavevmode\vadjust pre{\hypertarget{ref-pawitan_all_2001}{}}%
Pawitan, Y. (2001). \emph{In all likelihood: Statistical modelling and
inference using likelihood}. {Clarendon Press ; Oxford University
Press}.

\leavevmode\vadjust pre{\hypertarget{ref-R-patchwork}{}}%
Pedersen, T. L. (2020). \emph{Patchwork: The composer of plots}.
\url{https://CRAN.R-project.org/package=patchwork}

\leavevmode\vadjust pre{\hypertarget{ref-perneger_whats_1998}{}}%
Perneger, T. V. (1998). What's wrong with {Bonferroni} adjustments.
\emph{Bmj}, \emph{316}(7139), 1236--1238.

\leavevmode\vadjust pre{\hypertarget{ref-perugini_practical_2018}{}}%
Perugini, M., Gallucci, M., \& Costantini, G. (2018). A {Practical
Primer To Power Analysis} for {Simple Experimental Designs}.
\emph{International Review of Social Psychology}, \emph{31}(1), 20.
\url{https://doi.org/10.5334/irsp.181}

\leavevmode\vadjust pre{\hypertarget{ref-perugini_safeguard_2014}{}}%
Perugini, M., Gallucci, M., \& Costantini, G. (2014). Safeguard power as
a protection against imprecise power estimates. \emph{Perspectives on
Psychological Science}, \emph{9}(3), 319--332.
\url{https://doi.org/10.1177/1745691614528519}

\leavevmode\vadjust pre{\hypertarget{ref-peters_performance_2007}{}}%
Peters, J. L., Sutton, A. J., Jones, D. R., Abrams, K. R., \& Rushton,
L. (2007). Performance of the trim and fill method in the presence of
publication bias and between-study heterogeneity. \emph{Statistics in
Medicine}, \emph{26}(25), 4544--4562.
\url{https://doi.org/10.1002/sim.2889}

\leavevmode\vadjust pre{\hypertarget{ref-phillips_statistical_2001}{}}%
Phillips, B. M., Hunt, J. W., Anderson, B. S., Puckett, H. M., Fairey,
R., Wilson, C. J., \& Tjeerdema, R. (2001). Statistical significance of
sediment toxicity test results: {Threshold} values derived by the
detectable significance approach. \emph{Environmental Toxicology and
Chemistry}, \emph{20}(2), 371--373.
\url{https://doi.org/10.1002/etc.5620200218}

\leavevmode\vadjust pre{\hypertarget{ref-pocock_group_1977}{}}%
Pocock, S. J. (1977). Group sequential methods in the design and
analysis of clinical trials. \emph{Biometrika}, \emph{64}(2), 191--199.
\url{https://doi.org/10.1093/biomet/64.2.191}

\leavevmode\vadjust pre{\hypertarget{ref-polanin_transparency_2020}{}}%
Polanin, J. R., Hennessy, E. A., \& Tsuji, S. (2020). Transparency and
{Reproducibility} of {Meta-Analyses} in {Psychology}: {A Meta-Review}.
\emph{Perspectives on Psychological Science}, \emph{15}(4), 1026--1041.
\url{https://doi.org/10.1177/1745691620906416}

\leavevmode\vadjust pre{\hypertarget{ref-popper_logic_2002}{}}%
Popper, K. R. (2002). \emph{{The logic of scientific discovery}}.
{Routledge}.

\leavevmode\vadjust pre{\hypertarget{ref-primbs_are_2022}{}}%
Primbs, M., Pennington, C. R., Lakens, D., Silan, M. A., Lieck, D. S.
N., Forscher, P., Buchanan, E. M., \& Westwood, S. J. (2022). Are {Small
Effects} the {Indispensable Foundation} for a {Cumulative Psychological
Science}? {A Reply} to {Götz} et al. (2022). \emph{Perspectives on
Psychological Science}. \url{https://doi.org/10.31234/osf.io/6s8bj}

\leavevmode\vadjust pre{\hypertarget{ref-proschan_two-stage_2005}{}}%
Proschan, M. A. (2005). Two-{Stage Sample Size Re-Estimation Based} on a
{Nuisance Parameter}: {A Review}. \emph{Journal of Biopharmaceutical
Statistics}, \emph{15}(4), 559--574.
\url{https://doi.org/10.1081/BIP-200062852}

\leavevmode\vadjust pre{\hypertarget{ref-proschan_statistical_2006}{}}%
Proschan, M. A., Lan, K. K. G., \& Wittes, J. T. (2006).
\emph{Statistical monitoring of clinical trials: A unified approach}.
{Springer}.

\leavevmode\vadjust pre{\hypertarget{ref-quertemont_how_2011}{}}%
Quertemont, E. (2011). How to {Statistically Show} the {Absence} of an
{Effect}. \emph{Psychologica Belgica}, \emph{51}(2), 109--127.
\url{https://doi.org/10.5334/pb-51-2-109}

\leavevmode\vadjust pre{\hypertarget{ref-R-base}{}}%
R Core Team. (2021). \emph{R: A language and environment for statistical
computing}. R Foundation for Statistical Computing.
\url{https://www.R-project.org/}

\leavevmode\vadjust pre{\hypertarget{ref-rice_heads_1994}{}}%
Rice, W. R., \& Gaines, S. D. (1994). '{Heads I} win, tails you lose':
Testing directional alternative hypotheses in ecological and
evolutionary research. \emph{Trends in Ecology \& Evolution},
\emph{9}(6), 235--237.
\url{https://doi.org/10.1016/0169-5347(94)90258-5}

\leavevmode\vadjust pre{\hypertarget{ref-richard_one_2003}{}}%
Richard, F. D., Bond, C. F., \& Stokes-Zoota, J. J. (2003). One {Hundred
Years} of {Social Psychology Quantitatively Described}. \emph{Review of
General Psychology}, \emph{7}(4), 331--363.
\url{https://doi.org/10.1037/1089-2680.7.4.331}

\leavevmode\vadjust pre{\hypertarget{ref-richardson_eta_2011}{}}%
Richardson, J. T. E. (2011). Eta squared and partial eta squared as
measures of effect size in educational research. \emph{Educational
Research Review}, \emph{6}(2), 135--147.
\url{https://doi.org/10.1016/j.edurev.2010.12.001}

\leavevmode\vadjust pre{\hypertarget{ref-rijnsoever_i_2017}{}}%
Rijnsoever, F. J. van. (2017). ({I Can}'t {Get No}) {Saturation}: {A}
simulation and guidelines for sample sizes in qualitative research.
\emph{PLOS ONE}, \emph{12}(7), e0181689.
\url{https://doi.org/10.1371/journal.pone.0181689}

\leavevmode\vadjust pre{\hypertarget{ref-R-MASS}{}}%
Ripley, B. (2021). \emph{MASS: Support functions and datasets for
venables and ripley's MASS}. \url{http://www.stats.ox.ac.uk/pub/MASS4/}

\leavevmode\vadjust pre{\hypertarget{ref-rogers_using_1993}{}}%
Rogers, J. L., Howard, K. I., \& Vessey, J. T. (1993). Using
significance tests to evaluate equivalence between two experimental
groups. \emph{Psychological Bulletin}, \emph{113}(3), 553--565.
https://doi.org/\url{http://dx.doi.org/10.1037/0033-2909.113.3.553}

\leavevmode\vadjust pre{\hypertarget{ref-rogers_how_1992}{}}%
Rogers, S. (1992). How a publicity blitz created the myth of subliminal
advertising. \emph{Public Relations Quarterly}, \emph{37}(4), 12.

\leavevmode\vadjust pre{\hypertarget{ref-ropovik_neglect_2021}{}}%
Ropovik, I., Adamkovic, M., \& Greger, D. (2021). Neglect of publication
bias compromises meta-analyses of educational research. \emph{PLOS ONE},
\emph{16}(6), e0252415.
\url{https://doi.org/10.1371/journal.pone.0252415}

\leavevmode\vadjust pre{\hypertarget{ref-rosenthal_contrasts_2000}{}}%
Rosenthal, R., Rosnow, R. L., \& Rubin, D. B. (2000). \emph{Contrasts
and effect sizes in behavioral research: A correlational approach}.
{Cambridge University Press}.

\leavevmode\vadjust pre{\hypertarget{ref-rouder_bayesian_2009}{}}%
Rouder, J. N., Speckman, P. L., Sun, D., Morey, R. D., \& Iverson, G.
(2009). Bayesian t tests for accepting and rejecting the null
hypothesis. \emph{Psychonomic Bulletin \& Review}, \emph{16}(2),
225--237. \url{https://doi.org/10.3758/PBR.16.2.225}

\leavevmode\vadjust pre{\hypertarget{ref-royall_statistical_1997}{}}%
Royall, R. (1997). \emph{Statistical {Evidence}: {A Likelihood
Paradigm}}. {Chapman and Hall/CRC}.

\leavevmode\vadjust pre{\hypertarget{ref-rozeboom_fallacy_1960}{}}%
Rozeboom, W. W. (1960). The fallacy of the null-hypothesis significance
test. \emph{Psychological Bulletin}, \emph{57}(5), 416--428.
\url{https://doi.org/10.1037/h0042040}

\leavevmode\vadjust pre{\hypertarget{ref-rucker_undue_2008}{}}%
Rücker, G., Schwarzer, G., Carpenter, J. R., \& Schumacher, M. (2008).
Undue reliance on {I}(2) in assessing heterogeneity may mislead.
\emph{BMC Medical Research Methodology}, \emph{8}, 79.
\url{https://doi.org/10.1186/1471-2288-8-79}

\leavevmode\vadjust pre{\hypertarget{ref-sarafoglou_survey_2022}{}}%
Sarafoglou, A., Kovacs, M., Bakos, B., Wagenmakers, E.-J., \& Aczel, B.
(2022). A survey on how preregistration affects the research workflow:
Better science but more work. \emph{Royal Society Open Science},
\emph{9}(7), 211997. \url{https://doi.org/10.1098/rsos.211997}

\leavevmode\vadjust pre{\hypertarget{ref-scheel_excess_2021}{}}%
Scheel, A. M., Schijen, M. R. M. J., \& Lakens, D. (2021). An {Excess}
of {Positive Results}: {Comparing} the {Standard Psychology Literature
With Registered Reports}. \emph{Advances in Methods and Practices in
Psychological Science}, \emph{4}(2), 25152459211007467.
\url{https://doi.org/10.1177/25152459211007467}

\leavevmode\vadjust pre{\hypertarget{ref-scheel_why_2021}{}}%
Scheel, A. M., Tiokhin, L., Isager, P. M., \& Lakens, D. (2021). Why
{Hypothesis Testers Should Spend Less Time Testing Hypotheses}.
\emph{Perspectives on Psychological Science}, \emph{16}(4), 744--755.
\url{https://doi.org/10.1177/1745691620966795}

\leavevmode\vadjust pre{\hypertarget{ref-schimmack_ironic_2012}{}}%
Schimmack, U. (2012). The ironic effect of significant results on the
credibility of multiple-study articles. \emph{Psychological Methods},
\emph{17}(4), 551--566. \url{https://doi.org/10.1037/a0029487}

\leavevmode\vadjust pre{\hypertarget{ref-schnuerch_controlling_2020}{}}%
Schnuerch, M., \& Erdfelder, E. (2020). Controlling decision errors with
minimal costs: {The} sequential probability ratio t test.
\emph{Psychological Methods}, \emph{25}(2), 206--226.
\url{https://doi.org/10.1037/met0000234}

\leavevmode\vadjust pre{\hypertarget{ref-schoemann_determining_2017}{}}%
Schoemann, A. M., Boulton, A. J., \& Short, S. D. (2017). Determining
{Power} and {Sample Size} for {Simple} and {Complex Mediation Models}.
\emph{Social Psychological and Personality Science}, \emph{8}(4),
379--386. \url{https://doi.org/10.1177/1948550617715068}

\leavevmode\vadjust pre{\hypertarget{ref-schonbrodt_sequential_2017}{}}%
Schönbrodt, F. D., Wagenmakers, E.-J., Zehetleitner, M., \& Perugini, M.
(2017). Sequential hypothesis testing with {Bayes} factors:
{Efficiently} testing mean differences. \emph{Psychological Methods},
\emph{22}(2), 322--339. \url{https://doi.org/10.1037/MET0000061}

\leavevmode\vadjust pre{\hypertarget{ref-schuirmann_comparison_1987}{}}%
Schuirmann, D. J. (1987). A comparison of the two one-sided tests
procedure and the power approach for assessing the equivalence of
average bioavailability. \emph{Journal of Pharmacokinetics and
Biopharmaceutics}, \emph{15}(6), 657--680.

\leavevmode\vadjust pre{\hypertarget{ref-schulz_sample_2005}{}}%
Schulz, K. F., \& Grimes, D. A. (2005). Sample size calculations in
randomised trials: Mandatory and mystical. \emph{The Lancet},
\emph{365}(9467), 1348--1353.
\url{https://doi.org/10.1016/S0140-6736(05)61034-3}

\leavevmode\vadjust pre{\hypertarget{ref-schumi_through_2011}{}}%
Schumi, J., \& Wittes, J. T. (2011). Through the looking glass:
Understanding non-inferiority. \emph{Trials}, \emph{12}(1), 106.
\url{https://doi.org/10.1186/1745-6215-12-106}

\leavevmode\vadjust pre{\hypertarget{ref-schweder_confidence_2016}{}}%
Schweder, T., \& Hjort, N. L. (2016). \emph{Confidence, {Likelihood},
{Probability}: {Statistical Inference} with {Confidence Distributions}}.
{Cambridge University Press}.
\url{https://doi.org/10.1017/CBO9781139046671}

\leavevmode\vadjust pre{\hypertarget{ref-seaman_equivalence_1998}{}}%
Seaman, M. A., \& Serlin, R. C. (1998). Equivalence confidence intervals
for two-group comparisons of means. \emph{Psychological Methods},
\emph{3}(4), 403--411.
https://doi.org/\url{http://dx.doi.org.dianus.libr.tue.nl/10.1037/1082-989X.3.4.403}

\leavevmode\vadjust pre{\hypertarget{ref-sedlmeier_studies_1989}{}}%
Sedlmeier, P., \& Gigerenzer, G. (1989). Do studies of statistical power
have an effect on the power of studies? \emph{Psychological Bulletin},
\emph{105}(2), 309--316.
\url{https://doi.org/10.1037/0033-2909.105.2.309}

\leavevmode\vadjust pre{\hypertarget{ref-shmueli_explain_2010}{}}%
Shmueli, G. (2010). To explain or to predict? \emph{Statistical
Science}, \emph{25}(3), 289--310.

\leavevmode\vadjust pre{\hypertarget{ref-simmons_life_2013}{}}%
Simmons, J. P., Nelson, L. D., \& Simonsohn, U. (2013). \emph{Life after
{P-Hacking}}.

\leavevmode\vadjust pre{\hypertarget{ref-simmons_false-positive_2011}{}}%
Simmons, J. P., Nelson, L. D., \& Simonsohn, U. (2011). False-{Positive
Psychology}: {Undisclosed Flexibility} in {Data Collection} and
{Analysis Allows Presenting Anything} as {Significant}.
\emph{Psychological Science}, \emph{22}(11), 1359--1366.
\url{https://doi.org/10.1177/0956797611417632}

\leavevmode\vadjust pre{\hypertarget{ref-simonsohn_small_2015}{}}%
Simonsohn, U. (2015). Small telescopes: {Detectability} and the
evaluation of replication results. \emph{Psychological Science},
\emph{26}(5), 559--569. \url{https://doi.org/10.1177/0956797614567341}

\leavevmode\vadjust pre{\hypertarget{ref-simonsohn_p-curve_2014}{}}%
Simonsohn, U., Nelson, L. D., \& Simmons, J. P. (2014). P-curve: {A} key
to the file-drawer. \emph{Journal of Experimental Psychology: General},
\emph{143}(2), 534.

\leavevmode\vadjust pre{\hypertarget{ref-smaldino_natural_2016}{}}%
Smaldino, P. E., \& McElreath, R. (2016). The natural selection of bad
science. \emph{Royal Society Open Science}, \emph{3}(9), 160384.
\url{https://doi.org/10.1098/rsos.160384}

\leavevmode\vadjust pre{\hypertarget{ref-smithson_confidence_2003}{}}%
Smithson, M. (2003). \emph{Confidence intervals}. {Sage Publications}.

\leavevmode\vadjust pre{\hypertarget{ref-sotola_garbage_2022}{}}%
Sotola, L. K. (2022). Garbage {In}, {Garbage Out}? {Evaluating} the
{Evidentiary Value} of {Published Meta-analyses Using Z-Curve Analysis}.
\emph{Collabra: Psychology}, \emph{8}(1), 32571.
\url{https://doi.org/10.1525/collabra.32571}

\leavevmode\vadjust pre{\hypertarget{ref-spanos_who_2013}{}}%
Spanos, A. (2013). Who should be afraid of the {Jeffreys-Lindley}
paradox? \emph{Philosophy of Science}, \emph{80}(1), 73--93.
\url{https://doi.org/10.1086/668875}

\leavevmode\vadjust pre{\hypertarget{ref-spellman_short_2015}{}}%
Spellman, B. A. (2015). A {Short} ({Personal}) {Future History} of
{Revolution} 2.0. \emph{Perspectives on Psychological Science},
\emph{10}(6), 886--899. \url{https://doi.org/10.1177/1745691615609918}

\leavevmode\vadjust pre{\hypertarget{ref-spiegelhalter_art_2019}{}}%
Spiegelhalter, D. (2019). \emph{The {Art} of {Statistics}: {How} to
{Learn} from {Data}} (Illustrated edition). {Basic Books}.

\leavevmode\vadjust pre{\hypertarget{ref-spiegelhalter_monitoring_1986}{}}%
Spiegelhalter, D. J., Freedman, L. S., \& Blackburn, P. R. (1986).
Monitoring clinical trials: Conditional or predictive power?
\emph{Controlled Clinical Trials}, \emph{7}(1), 8--17.
\url{https://doi.org/10.1016/0197-2456(86)90003-6}

\leavevmode\vadjust pre{\hypertarget{ref-stanley_meta-regression_2014}{}}%
Stanley, T. D., \& Doucouliagos, H. (2014). Meta-regression
approximations to reduce publication selection bias. \emph{Research
Synthesis Methods}, \emph{5}(1), 60--78.
\url{https://doi.org/10.1002/jrsm.1095}

\leavevmode\vadjust pre{\hypertarget{ref-stanley_finding_2017}{}}%
Stanley, T. D., Doucouliagos, H., \& Ioannidis, J. P. A. (2017). Finding
the power to reduce publication bias: {Finding} the power to reduce
publication bias. \emph{Statistics in Medicine}.
\url{https://doi.org/10.1002/sim.7228}

\leavevmode\vadjust pre{\hypertarget{ref-steiger_beyond_2004}{}}%
Steiger, J. H. (2004). Beyond the {F Test}: {Effect Size Confidence
Intervals} and {Tests} of {Close Fit} in the {Analysis} of {Variance}
and {Contrast Analysis}. \emph{Psychological Methods}, \emph{9}(2),
164--182. \url{https://doi.org/10.1037/1082-989X.9.2.164}

\leavevmode\vadjust pre{\hypertarget{ref-sterling_publication_1959}{}}%
Sterling, T. D. (1959). Publication {Decisions} and {Their Possible
Effects} on {Inferences Drawn} from {Tests} of {Significance--Or Vice
Versa}. \emph{Journal of the American Statistical Association},
\emph{54}(285), 30--34. \url{https://doi.org/10.2307/2282137}

\leavevmode\vadjust pre{\hypertarget{ref-stewart_ipd_2002}{}}%
Stewart, L. A., \& Tierney, J. F. (2002). To {IPD} or not to {IPD}?:
{Advantages} and {Disadvantages} of {Systematic Reviews Using Individual
Patient Data}. \emph{Evaluation \& the Health Professions},
\emph{25}(1), 76--97. \url{https://doi.org/10.1177/0163278702025001006}

\leavevmode\vadjust pre{\hypertarget{ref-taper_philosophy_2011}{}}%
Taper, M. L., \& Lele, S. R. (2011). Philosophy of {Statistics}. In P.
S. Bandyophadhyay \& M. R. Forster (Eds.), \emph{Evidence, evidence
functions, and error probabilities} (pp. 513--531). {Elsevier, USA}.

\leavevmode\vadjust pre{\hypertarget{ref-taylor_bias_1996}{}}%
Taylor, D. J., \& Muller, K. E. (1996). Bias in linear model power and
sample size calculation due to estimating noncentrality.
\emph{Communications in Statistics-Theory and Methods}, \emph{25}(7),
1595--1610. \url{https://doi.org/10.1080/03610929608831787}

\leavevmode\vadjust pre{\hypertarget{ref-teare_sample_2014}{}}%
Teare, M. D., Dimairo, M., Shephard, N., Hayman, A., Whitehead, A., \&
Walters, S. J. (2014). Sample size requirements to estimate key design
parameters from external pilot randomised controlled trials: A
simulation study. \emph{Trials}, \emph{15}(1), 264.
\url{https://doi.org/10.1186/1745-6215-15-264}

\leavevmode\vadjust pre{\hypertarget{ref-tendeiro_review_2019}{}}%
Tendeiro, J. N., \& Kiers, H. A. L. (2019). A review of issues about
null hypothesis {Bayesian} testing. \emph{Psychological Methods}.
\url{https://doi.org/10.1037/met0000221}

\leavevmode\vadjust pre{\hypertarget{ref-ter_schure_accumulation_2019}{}}%
ter Schure, J., \& Grünwald, P. D. (2019). Accumulation {Bias} in
{Meta-Analysis}: {The Need} to {Consider Time} in {Error Control}.
\emph{arXiv:1905.13494 {[}Math, Stat{]}}.
\url{https://arxiv.org/abs/1905.13494}

\leavevmode\vadjust pre{\hypertarget{ref-terrin_adjusting_2003}{}}%
Terrin, N., Schmid, C. H., Lau, J., \& Olkin, I. (2003). Adjusting for
publication bias in the presence of heterogeneity. \emph{Statistics in
Medicine}, \emph{22}(13), 2113--2126.
\url{https://doi.org/10.1002/sim.1461}

\leavevmode\vadjust pre{\hypertarget{ref-thompson_effect_2007}{}}%
Thompson, B. (2007). Effect sizes, confidence intervals, and confidence
intervals for effect sizes. \emph{Psychology in the Schools},
\emph{44}(5), 423--432. \url{https://doi.org/10.1002/pits.20234}

\leavevmode\vadjust pre{\hypertarget{ref-tversky_features_1977}{}}%
Tversky, A. (1977). Features of similarity. \emph{Psychological Review},
\emph{84}(4), 327--352. \url{https://doi.org/10.1037/0033-295X.84.4.327}

\leavevmode\vadjust pre{\hypertarget{ref-tversky_belief_1971}{}}%
Tversky, A., \& Kahneman, D. (1971). Belief in the law of small numbers.
\emph{Psychological Bulletin}, \emph{76}(2), 105--110.
\url{https://doi.org/10.1037/h0031322}

\leavevmode\vadjust pre{\hypertarget{ref-ulrich_properties_2018}{}}%
Ulrich, R., \& Miller, J. (2018). Some properties of p-curves, with an
application to gradual publication bias. \emph{Psychological Methods},
\emph{23}(3), 546--560. \url{https://doi.org/10.1037/met0000125}

\leavevmode\vadjust pre{\hypertarget{ref-uygun_tunc_falsificationist_2022}{}}%
Uygun Tunç, D., \& Tunç, M. N. (2022). A {Falsificationist Treatment} of
{Auxiliary Hypotheses} in {Social} and {Behavioral Sciences}:
{Systematic Replications Framework}. \emph{Meta-Psychology}.
\url{https://doi.org/10.31234/osf.io/pdm7y}

\leavevmode\vadjust pre{\hypertarget{ref-uygun_tunc_epistemic_2021}{}}%
Uygun Tunç, D., Tunç, M. N., \& Lakens, D. (2021). \emph{The {Epistemic}
and {Pragmatic Function} of {Dichotomous Claims Based} on {Statistical
Hypothesis Tests}}. {PsyArXiv}.
\url{https://doi.org/10.31234/osf.io/af9by}

\leavevmode\vadjust pre{\hypertarget{ref-valentine_how_2010}{}}%
Valentine, J. C., Pigott, T. D., \& Rothstein, H. R. (2010). How {Many
Studies Do You Need}?: {A Primer} on {Statistical Power} for
{Meta-Analysis}. \emph{Journal of Educational and Behavioral
Statistics}, \emph{35}(2), 215--247.
\url{https://doi.org/10.3102/1076998609346961}

\leavevmode\vadjust pre{\hypertarget{ref-R-puniform}{}}%
van Aert, R. C. M. (2021). \emph{Puniform: Meta-analysis methods
correcting for publication bias}.
\url{https://github.com/RobbievanAert/puniform}

\leavevmode\vadjust pre{\hypertarget{ref-van_de_schoot_use_2021}{}}%
van de Schoot, R., Winter, S. D., Griffioen, E., Grimmelikhuijsen, S.,
Arts, I., Veen, D., Grandfield, E. M., \& Tummers, L. G. (2021). The
{Use} of {Questionable Research Practices} to {Survive} in {Academia
Examined With Expert Elicitation}, {Prior-Data Conflicts}, {Bayes
Factors} for {Replication Effects}, and the {Bayes Truth Serum}.
\emph{Frontiers in Psychology}, \emph{12}.

\leavevmode\vadjust pre{\hypertarget{ref-van_t_veer_pre-registration_2016}{}}%
van 't Veer, A. E., \& Giner-Sorolla, R. (2016). Pre-registration in
social psychology\textemdash{{A}} discussion and suggested template.
\emph{Journal of Experimental Social Psychology}, \emph{67}, 2--12.
\url{https://doi.org/10.1016/j.jesp.2016.03.004}

\leavevmode\vadjust pre{\hypertarget{ref-MASS2002}{}}%
Venables, W. N., \& Ripley, B. D. (2002). \emph{Modern applied
statistics with s} (Fourth). Springer.
\url{https://www.stats.ox.ac.uk/pub/MASS4/}

\leavevmode\vadjust pre{\hypertarget{ref-verschuere_registered_2018}{}}%
Verschuere, B., Meijer, E. H., Jim, A., Hoogesteyn, K., Orthey, R.,
McCarthy, R. J., Skowronski, J. J., Acar, O. A., Aczel, B., Bakos, B.
E., Barbosa, F., Baskin, E., Bègue, L., Ben-Shakhar, G., Birt, A. R.,
Blatz, L., Charman, S. D., Claesen, A., Clay, S. L., \ldots{} Yıldız, E.
(2018). Registered {Replication Report} on {Mazar}, {Amir}, and {Ariely}
(2008). \emph{Advances in Methods and Practices in Psychological
Science}, \emph{1}(3), 299--317.
\url{https://doi.org/10.1177/2515245918781032}

\leavevmode\vadjust pre{\hypertarget{ref-viamonte_cost-benefit_2006}{}}%
Viamonte, S. M., Ball, K. K., \& Kilgore, M. (2006). A {Cost-Benefit
Analysis} of {Risk-Reduction Strategies Targeted} at {Older Drivers}.
\emph{Traffic Injury Prevention}, \emph{7}(4), 352--359.
\url{https://doi.org/10.1080/15389580600791362}

\leavevmode\vadjust pre{\hypertarget{ref-viechtbauer_conducting_2010}{}}%
Viechtbauer, W. (2010). Conducting meta-analyses in {R} with the metafor
package. \emph{J Stat Softw}, \emph{36}(3), 1--48.
https://doi.org/\url{http://dx.doi.org/10.18637/jss.v036.i03}

\leavevmode\vadjust pre{\hypertarget{ref-R-metafor}{}}%
Viechtbauer, W. (2021). \emph{Metafor: Meta-analysis package for r}.
\url{https://CRAN.R-project.org/package=metafor}

\leavevmode\vadjust pre{\hypertarget{ref-vohs_multisite_2021}{}}%
Vohs, K. D., Schmeichel, B. J., Lohmann, S., Gronau, Q. F., Finley, A.
J., Ainsworth, S. E., Alquist, J. L., Baker, M. D., Brizi, A., Bunyi,
A., Butschek, G. J., Campbell, C., Capaldi, J., Cau, C., Chambers, H.,
Chatzisarantis, N. L. D., Christensen, W. J., Clay, S. L., Curtis, J.,
\ldots{} Albarracín, D. (2021). A {Multisite Preregistered Paradigmatic
Test} of the {Ego-Depletion Effect}. \emph{Psychological Science},
\emph{32}(10), 1566--1581.
\url{https://doi.org/10.1177/0956797621989733}

\leavevmode\vadjust pre{\hypertarget{ref-vosgerau_99_2019}{}}%
Vosgerau, J., Simonsohn, U., Nelson, L. D., \& Simmons, J. P. (2019).
99\% impossible: {A} valid, or falsifiable, internal meta-analysis.
\emph{Journal of Experimental Psychology. General}, \emph{148}(9),
1628--1639. \url{https://doi.org/10.1037/xge0000663}

\leavevmode\vadjust pre{\hypertarget{ref-vuorre_curating_2018}{}}%
Vuorre, M., \& Curley, J. P. (2018). Curating {Research Assets}: {A
Tutorial} on the {Git Version Control System}. \emph{Advances in Methods
and Practices in Psychological Science}, \emph{1}(2), 219--236.
\url{https://doi.org/10.1177/2515245918754826}

\leavevmode\vadjust pre{\hypertarget{ref-wacholder_assessing_2004}{}}%
Wacholder, S., Chanock, S., Garcia-Closas, M., El ghormli, L., \&
Rothman, N. (2004). Assessing the {Probability That} a {Positive Report}
is {False}: {An Approach} for {Molecular Epidemiology Studies}.
\emph{JNCI Journal of the National Cancer Institute}, \emph{96}(6),
434--442. \url{https://doi.org/10.1093/jnci/djh075}

\leavevmode\vadjust pre{\hypertarget{ref-wagenmakers_practical_2007}{}}%
Wagenmakers, E.-J. (2007). A practical solution to the pervasive
problems of p values. \emph{Psychonomic Bulletin \& Review},
\emph{14}(5), 779--804. \url{https://doi.org/10.3758/BF03194105}

\leavevmode\vadjust pre{\hypertarget{ref-wagenmakers_registered_2016}{}}%
Wagenmakers, E.-J., Beek, T., Dijkhoff, L., Gronau, Q. F., Acosta, A.,
Adams, R. B., Albohn, D. N., Allard, E. S., Benning, S. D.,
Blouin-Hudon, E.-M., Bulnes, L. C., Caldwell, T. L., Calin-Jageman, R.
J., Capaldi, C. A., Carfagno, N. S., Chasten, K. T., Cleeremans, A.,
Connell, L., DeCicco, J. M., \ldots{} Zwaan, R. A. (2016). Registered
{Replication Report}: {Strack}, {Martin}, \& {Stepper} (1988).
\emph{Perspectives on Psychological Science}, \emph{11}(6), 917--928.
\url{https://doi.org/10.1177/1745691616674458}

\leavevmode\vadjust pre{\hypertarget{ref-wald_sequential_1945}{}}%
Wald, A. (1945). Sequential tests of statistical hypotheses. \emph{The
Annals of Mathematical Statistics}, \emph{16}(2), 117--186.
https://doi.org/\url{https://www.jstor.org/stable/2240273}

\leavevmode\vadjust pre{\hypertarget{ref-wassmer_group_2016}{}}%
Wassmer, G., \& Brannath, W. (2016). \emph{Group {Sequential} and
{Confirmatory Adaptive Designs} in {Clinical Trials}}. {Springer
International Publishing}.
\url{https://doi.org/10.1007/978-3-319-32562-0}

\leavevmode\vadjust pre{\hypertarget{ref-wassmer_rpact_2019}{}}%
Wassmer, G., \& Pahlke, F. (2019). \emph{Rpact: {Confirmatory} adaptive
clinical trial design and analysis}.

\leavevmode\vadjust pre{\hypertarget{ref-R-rpact}{}}%
Wassmer, G., \& Pahlke, F. (2022). \emph{Rpact: Confirmatory adaptive
clinical trial design and analysis}. \url{https://www.rpact.org}

\leavevmode\vadjust pre{\hypertarget{ref-weinshall-margel_overlooked_2011}{}}%
Weinshall-Margel, K., \& Shapard, J. (2011). Overlooked factors in the
analysis of parole decisions. \emph{Proceedings of the National Academy
of Sciences}, \emph{108}(42), E833--E833.
\url{https://doi.org/10.1073/pnas.1110910108}

\leavevmode\vadjust pre{\hypertarget{ref-wellek_testing_2010}{}}%
Wellek, S. (2010). \emph{Testing statistical hypotheses of equivalence
and noninferiority} (2nd ed). {CRC Press}.

\leavevmode\vadjust pre{\hypertarget{ref-westberg_combining_1985}{}}%
Westberg, M. (1985). Combining {Independent Statistical Tests}.
\emph{Journal of the Royal Statistical Society. Series D (The
Statistician)}, \emph{34}(3), 287--296.
\url{https://doi.org/10.2307/2987655}

\leavevmode\vadjust pre{\hypertarget{ref-westfall_statistical_2014}{}}%
Westfall, J., Kenny, D. A., \& Judd, C. M. (2014). Statistical power and
optimal design in experiments in which samples of participants respond
to samples of stimuli. \emph{Journal of Experimental Psychology:
General}, \emph{143}(5), 2020--2045.
\url{https://doi.org/10.1037/xge0000014}

\leavevmode\vadjust pre{\hypertarget{ref-westlake_use_1972}{}}%
Westlake, W. J. (1972). Use of {Confidence Intervals} in {Analysis} of
{Comparative Bioavailability Trials}. \emph{Journal of Pharmaceutical
Sciences}, \emph{61}(8), 1340--1341.
\url{https://doi.org/10.1002/JPS.2600610845}

\leavevmode\vadjust pre{\hypertarget{ref-wicherts_degrees_2016}{}}%
Wicherts, J. M., Veldkamp, C. L. S., Augusteijn, H. E. M., Bakker, M.,
Aert, V., M, R. C., Assen, V., \& M, M. A. L. (2016). Degrees of
{Freedom} in {Planning}, {Running}, {Analyzing}, and {Reporting
Psychological Studies}: {A Checklist} to {Avoid} p-{Hacking}.
\emph{Frontiers in Psychology}, \emph{7}.
\url{https://doi.org/10.3389/fpsyg.2016.01832}

\leavevmode\vadjust pre{\hypertarget{ref-ggplot22016}{}}%
Wickham, H. (2016). \emph{ggplot2: Elegant graphics for data analysis}.
Springer-Verlag New York. \url{https://ggplot2.tidyverse.org}

\leavevmode\vadjust pre{\hypertarget{ref-R-tidyverse}{}}%
Wickham, H. (2021). \emph{Tidyverse: Easily install and load the
tidyverse}. \url{https://CRAN.R-project.org/package=tidyverse}

\leavevmode\vadjust pre{\hypertarget{ref-wiebels_leveraging_2021}{}}%
Wiebels, K., \& Moreau, D. (2021). Leveraging {Containers} for
{Reproducible Psychological Research}. \emph{Advances in Methods and
Practices in Psychological Science}, \emph{4}(2), 25152459211017853.
\url{https://doi.org/10.1177/25152459211017853}

\leavevmode\vadjust pre{\hypertarget{ref-williams_impact_1995}{}}%
Williams, R. H., Zimmerman, D. W., \& Zumbo, B. D. (1995). Impact of
{Measurement Error} on {Statistical Power}: {Review} of an {Old
Paradox}. \emph{The Journal of Experimental Education}, \emph{63}(4),
363--370. \url{https://doi.org/10.1080/00220973.1995.9943470}

\leavevmode\vadjust pre{\hypertarget{ref-wilson_practical_2015}{}}%
Wilson, E. C. F. (2015). A {Practical Guide} to {Value} of {Information
Analysis}. \emph{PharmacoEconomics}, \emph{33}(2), 105--121.
\url{https://doi.org/10.1007/s40273-014-0219-x}

\leavevmode\vadjust pre{\hypertarget{ref-wilson_vanvoorhis_understanding_2007}{}}%
Wilson VanVoorhis, C. R., \& Morgan, B. L. (2007). Understanding power
and rules of thumb for determining sample sizes. \emph{Tutorials in
Quantitative Methods for Psychology}, \emph{3}(2), 43--50.
\url{https://doi.org/10.20982/tqmp.03.2.p043}

\leavevmode\vadjust pre{\hypertarget{ref-winer_statistical_1962}{}}%
Winer, B. J. (1962). \emph{Statistical principles in experimental
design}. {New York : McGraw-Hill}.

\leavevmode\vadjust pre{\hypertarget{ref-wiseman_registered_2019}{}}%
Wiseman, R., Watt, C., \& Kornbrot, D. (2019). Registered reports: An
early example and analysis. \emph{PeerJ}, \emph{7}, e6232.
\url{https://doi.org/10.7717/peerj.6232}

\leavevmode\vadjust pre{\hypertarget{ref-wittes_role_1990}{}}%
Wittes, J., \& Brittain, E. (1990). The role of internal pilot studies
in increasing the efficiency of clinical trials. \emph{Statistics in
Medicine}, \emph{9}(1-2), 65--72.
\url{https://doi.org/10.1002/sim.4780090113}

\leavevmode\vadjust pre{\hypertarget{ref-wong_potential_2021}{}}%
Wong, T. K., Kiers, H., \& Tendeiro, J. (2021). \emph{On the {Potential
Mismatch} between the {Function} of the {Bayes Factor} and
{Researchers}' {Expectations}}. {PsyArXiv}.
\url{https://doi.org/10.31234/osf.io/86p4k}

\leavevmode\vadjust pre{\hypertarget{ref-wynants_prediction_2020}{}}%
Wynants, L., Calster, B. V., Collins, G. S., Riley, R. D., Heinze, G.,
Schuit, E., Bonten, M. M. J., Dahly, D. L., Damen, J. A., Debray, T. P.
A., Jong, V. M. T. de, Vos, M. D., Dhiman, P., Haller, M. C., Harhay, M.
O., Henckaerts, L., Heus, P., Kammer, M., Kreuzberger, N., \ldots{}
Smeden, M. van. (2020). Prediction models for diagnosis and prognosis of
covid-19: Systematic review and critical appraisal. \emph{BMJ},
\emph{369}, m1328. \url{https://doi.org/10.1136/bmj.m1328}

\leavevmode\vadjust pre{\hypertarget{ref-knitr2015}{}}%
Xie, Y. (2015). \emph{Dynamic documents with {R} and knitr} (2nd ed.).
Chapman; Hall/CRC. \url{https://yihui.org/knitr/}

\leavevmode\vadjust pre{\hypertarget{ref-R-bookdown}{}}%
Xie, Y. (2022). \emph{Bookdown: Authoring books and technical documents
with r markdown}. \url{https://CRAN.R-project.org/package=bookdown}

\leavevmode\vadjust pre{\hypertarget{ref-rmarkdown2020}{}}%
Xie, Y., Dervieux, C., \& Riederer, E. (2020). \emph{R markdown
cookbook}. Chapman; Hall/CRC.
\url{https://bookdown.org/yihui/rmarkdown-cookbook}

\leavevmode\vadjust pre{\hypertarget{ref-yarkoni_choosing_2017}{}}%
Yarkoni, T., \& Westfall, J. (2017). Choosing {Prediction Over
Explanation} in {Psychology}: {Lessons From Machine Learning}.
\emph{Perspectives on Psychological Science}, \emph{12}(6), 1100--1122.
\url{https://doi.org/10.1177/1745691617693393}

\leavevmode\vadjust pre{\hypertarget{ref-yuan_post_2005}{}}%
Yuan, K.-H., \& Maxwell, S. (2005). On the {Post Hoc Power} in {Testing
Mean Differences}. \emph{Journal of Educational and Behavioral
Statistics}, \emph{30}(2), 141--167.
\url{https://doi.org/10.3102/10769986030002141}

\leavevmode\vadjust pre{\hypertarget{ref-zabell_r_1992}{}}%
Zabell, S. L. (1992). R. {A}. {Fisher} and {Fiducial Argument}.
\emph{Statistical Science}, \emph{7}(3), 369--387.
\url{https://doi.org/10.1214/ss/1177011233}

\leavevmode\vadjust pre{\hypertarget{ref-R-kableExtra}{}}%
Zhu, H. (2021). \emph{kableExtra: Construct complex table with kable and
pipe syntax}. \url{https://CRAN.R-project.org/package=kableExtra}

\leavevmode\vadjust pre{\hypertarget{ref-zumbo_note_1998}{}}%
Zumbo, B. D., \& Hubley, A. M. (1998). A note on misconceptions
concerning prospective and retrospective power. \emph{Journal of the
Royal Statistical Society: Series D (The Statistician)}, \emph{47}(2),
385--388. \url{https://doi.org/10.1111/1467-9884.00139}

\end{CSLReferences}

\end{document}
